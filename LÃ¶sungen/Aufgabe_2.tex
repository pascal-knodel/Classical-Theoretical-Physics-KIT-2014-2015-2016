% !TeX encoding = UTF-8



\newpage



\chapter*{Übungsblatt 2}
\addcontentsline{toc}{chapter}{2}


\newpage


\section*{\underline{Fragen zu den Aufgaben oder Allgemeinem}}
\addcontentsline{toc}{subsubsection}{Fragen}

~\\

\begin{enumerate}
	
	\item Was sind die Stammfunktionen von ~ $\frac{1}{x}$ ~ ? \\
	
	\begin{tabularx}{0.88\textwidth}{lX}
		$\bullet$ & $\int ~ dy ~ \frac{1}{x} ~~ = ~~ ln ~ \left| x \right| ~ + ~ C \qquad, ~~ C ~ \in ~ \mathbb{K}$
	\end{tabularx}
	
	~\\
	
	\item Welche Beziehung gilt bei genau einer beliebigen stetigen Funktion die keine Nullstellen hat zwischen Vorzeichen von (mindestens zwei) ihrer beliebigen Parametrisierungen? \\
	
	\begin{tabularx}{0.88\textwidth}{lX}
		$\bullet$ & Sei ~ $\text{\underline{sf}}(x)$ ~ die beschriebene stetige Funktion und seien ~ $\text{\underline{sf}}\left(~ p_{~ i} ~\right)$ ~ Parametrisierungen mit ~ $i ~ \in ~ \{ i_1, ~ i_2 \} \subset ~ \mathbb{N}$ ~ . So gilt: \newline\newline ~ $sgn ~~ \text{\underline{sf}}\left(~ p_{~ i_1} ~\right) ~~ = ~~ sgn ~~ \text{\underline{sf}}\left(~ p_{~ i_2} ~\right)$ ~ .
	\end{tabularx}
	
	~\\
	
	\item Was ist zu tun, wenn in der Aufgabe steht: ~ \textquotedblleft~\textit{Drücken Sie ~ $F_1$ ~ durch ~ $F_2$ ~ aus.}~\textquotedblright ~ ? \\
	
	\begin{tabularx}{0.88\textwidth}{lX}
		$\bullet$ & Eine Gleichung ~ $... ~ F_2 ~ ... ~~ = ~~ ...$ ~ in die Form ~ $F_1 ~~ = ~~ ... ~ F_2 ~ ...$ ~ bringen.
	\end{tabularx}
	
%	~\\
%	
%	\item Der Ausdruck ~ $ \int_{~ x_1}^{x_2} ~ dx ~ f(x)$ ~ ist gegeben. Wie sind hier alle Stammfunktionen von der inneren Funktion des unbestimmten Integrals ~ $f(x)$ ~ formalisierbar? \\
%	
%	\begin{tabularx}{0.88\textwidth}{lX}
%		$\bullet$ & $ F(x) ~ + ~ C ~~ := ~~ \int ~dx ~ f(x) \qquad, ~~ C ~ \in ~ \mathbb{K} $ ~ . \newline Bestimmte und unbestimmte Integrale nicht verwechseln!
%	\end{tabularx}
	
	
	\newpage
	
	
	\item Warum ist ~ $ln ~ \left| x \right|$ ~ die Stammfunktion von ~ $\frac{1}{x}$ ~ ? \\
	
	\begin{tabularx}{0.88\textwidth}{lX}
		$\bullet$ & (TODO: Antwort)
	\end{tabularx}
	
	
	\newpage
	
	
	\item Ist dieser Beweis richtig  ... ~\\
	
	\begin{tabularx}{0.88\textwidth}{lX}
		Behauptung: & $F(x)$ ~ ist eine spezielle Lösung der Differentialgleichung ~ $\dydx ~ = ~ f(x) ~ y(x)$ ~ .
	\end{tabularx}
	 
	~\\
	~\\
	
	$ (*) ~ := ~ \begin{cases}
	~ F(x) ~~ &:= ~~ \int_{x_0}^{x_1} ~ dx ~ f(x) \\ \\
	~ y(x) ~~ &:= ~~ F(x)
	\end{cases}$ \\
	
	~\\
	
	$\begin{aligned}[t]
	(*) ~ &\Rightarrow ~~ \int_{x_0}^{x_1} ~ dx ~ \frac{1}{y(x)} ~ \dydx ~~ = ~~ \int_{x_0}^{x_1} ~ dx ~ f(x) \\ \\
	&\Leftrightarrow ~~ \int_{x_0}^{x_1} ~ dx ~ \frac{1}{F(x)} ~ \dFdx ~~ = ~~ F(x) \\ \\
	&\Leftrightarrow ~~ \int_{x_0}^{x_1} ~ dx ~ \frac{1}{F(x)} ~ f(x) ~~ = ~~ F(x) \\ \\
	&\Leftrightarrow ~~ \int_{x_0}^{x_1} ~ dx ~ \frac{1}{F(x)} ~ f(x) ~~ = ~~ \int_{x_0}^{x_1} ~ dx ~ f(x) \\ \\
	&\Leftrightarrow ~~ \int_{x_0}^{x_1} ~ dx ~ \frac{1}{F(x)} ~ f(x) ~~ = ~~ \int_{x_0}^{x_1} ~ dx ~ 1 ~ f(x) \\ \\
	&\Leftrightarrow ~~ \frac{1}{F(x)} ~~ = ~~ 1\\ \\
	&\Leftrightarrow ~~ F(x) ~~ = ~~ 1 \\ \\
	&\Leftrightarrow ~~ y(x) ~~ = ~~ 1
	\end{aligned}$
	
	
	\newpage
	
	
	Probe: ~ $y(x) ~ = ~ 1$ ~ :
	
	~\\
	
	$\begin{aligned}[t]
	&\textcolor{white}{\Leftrightarrow} ~~ \ddx ~ y(x) ~~ = ~~ f(x) ~  y(x) \\ \\
	&\Leftrightarrow ~~ \ddx ~ 1 ~~ = ~~ f(x) ~  1 \\ \\
	&\Leftrightarrow ~~ 0 ~~ = ~~ f(x) \\ \\
	&\Leftrightarrow ~~ wahr
	\end{aligned}$
	
	~\\
	~\\
	
	...?
	
	~\\
	
	\begin{tabularx}{0.88\textwidth}{lX}
		$\bullet$ & (TODO: Antwort)
	\end{tabularx}
	
	~\\

%	\item ...
	
	
\end{enumerate}


\newpage


\section*{Aufgabe 3}
%\addcontentsline{toc}{section}{7}


\newpage



\subsection*{3, a)}
\addcontentsline{toc}{section}{3, a)}


	\begin{longtable}[l]{rl}
		
		Forderung:  &  Anwendung von partieller Integration. \\
		
		\\
		
		  Bekannt:  &  $\ddy ~ e^y ~ = ~ e^y$.
		
	\end{longtable}
	
	~\\ $\begin{aligned}[t]
		I_0(x) ~~ &= ~~ \int_{0}^{x} ~ dy ~ y^0 ~ e^y \\ \\
		&= ~~ \int_{0}^{x} ~ dy ~ 1 ~ e^y \\ \\
		&= ~~ \int_{0}^{x} ~ dy ~ \underbrace{1}_{ \int } ~ \underbrace{ e^y }_{ \ddy } \\ \\
		&= ~~ \left[ ~ 1 ~ \left( \ddy ~ e^y \right) ~ \right]_{0}^{x} ~ - ~ \int_{0}^{x} ~ dy ~ \left( \ddy ~ 1 \right) ~ e^y \\ \\
		&= ~~ \left[ ~ e^y ~ \right]_{0}^{x} ~ - ~ \int_{0}^{x} ~ dy ~ 0 ~ e^y \\ \\
		&= ~~ e^x ~ - ~ e^0 ~ - ~ \int_{0}^{x} ~ dy ~ 0 \\ \\
		&= ~~ e^x ~ - ~ 1 ~ - ~ 0 \\ \\
		&= ~~ e^x ~ - ~ 1 \\ \\
	\end{aligned}$
	
	\newpage
	


\subsection*{3, b)}
\addcontentsline{toc}{section}{3, b)}
	
	$I_n(x) ~ = ~ \int_{0}^{x} ~ dy ~ y^n ~ e^y ~~ \xLeftrightarrow{\text{~ Partielle Integration ~}} ~~ \int_{0}^{x} ~ dy ~ y^{ ~ n ~ - ~ 1} ~ e^y ~ = ~ I_{~ (n ~ - ~ 1)}(x)$
	
	~\\
	
	$\begin{aligned}[t]
	I_n(x) ~~ &= ~~ \int_{0}^{x} ~ dy ~ \underbrace{ y^n }_{ \int } ~ \underbrace{ e^y }_{ \ddy } \\ \\
	&= ~~ \left[ y^n ~ \left( \ddy ~ e^y \right) \right]_{ 0 }^{ x } ~ - ~ \int_{0}^{x} ~ dy ~ \left( \ddy ~ y^n \right) ~ e^y \\ \\
	&= ~~ \left[ ~ y^n ~ e^y ~ \right]_{ 0 }^{ x } ~ - ~ \int_{0}^{x} ~ dy ~ n ~ y^{ ~ n ~ - ~ 1} ~ e^y \\ \\
	&= ~~ x^n ~ e^x ~ - ~ 0 ~ ... ~ - ~ n ~ \underbrace{ \int_{0}^{x} ~ dy ~ y^{ ~ n ~ - ~ 1} ~ e^y }_{ I_{~ (n ~ - ~ 1) ~}(x) } \\ \\
	&= ~~ x^n ~ e^x ~ - ~ n ~ I_{~ (n ~ - ~ 1) ~}(x) \\ \\
	\end{aligned}$ ~\\ 
	

\subsection*{3, c)}
\addcontentsline{toc}{section}{3, c)}
	
	$\begin{aligned}[t]
	I_1(x) ~~ &= ~~ x ~ e^x ~ - ~ 1 ~ I_{~ (1 ~ - ~ 1) ~}(x) \\ \\
	&= ~~ x ~ e^x ~ - ~ I_{0}(x) \\ \\
	&= ~~ x ~ e^x ~ - ~ I_{0}(x) \\ \\
	&= ~~ x ~ e^x ~ - ~ \left( e^x ~ - ~ 1 \right) \\ \\
	&= ~~ x ~ e^x ~ - ~ e^x ~ + ~ 1 \\ \\
	&= ~~ \left( x ~ - 1 \right) ~ e^x ~ + ~ 1 \\ \\
	\end{aligned}$
	
	\newpage
	
	$\begin{aligned}[t]
	I_2(x) ~~ &= ~~ x^2 ~ e^x ~ - ~ 2 ~ I_{~ (2 ~ - ~ 1) ~}(x) \\ \\
	&= ~~ x^2 ~ e^x ~ - ~ 2 ~ I_{1}(x) \\ \\
	&= ~~ x^2 ~ e^x ~ - ~ 2 ~ \left( ~ \left( x ~ - 1 \right) ~ e^x ~ + ~ 1 ~ \right) \\ \\
	&= ~~ x^2 ~ e^x ~ - ~ 2 ~  \left( x ~ - 1 \right) ~ e^x ~ - ~ 2 \\ \\
	&= ~~ \left( ~ x^2 ~ - ~ 2 ~ \left( x ~ - 1 \right) ~ \right) ~ e^x ~ - ~ 2 \\ \\
	&= ~~ \left( ~ x^2 ~ - ~ 2 ~ x ~ + 2 ~ \right) ~ e^x ~ - ~ 2 \\ \\
	\end{aligned}$ \\
	
	~\\
	
	$\begin{aligned}[t]
	I_3(x) ~~ &= ~~ x^3 ~ e^x ~ - ~ 3 ~ I_{~ (3 ~ - ~ 1) ~}(x) \\ \\
	&= ~~ x^3 ~ e^x ~ - ~ 3 ~ I_{2}(x) \\ \\
	&= ~~ x^3 ~ e^x ~ - ~ 3 ~ \left( ~ \left( ~ x^2 ~ - ~ 2 ~ x ~ + 2 ~ \right) ~ e^x ~ - ~ 2 ~ \right) \\ \\
	&= ~~ x^3 ~ e^x ~ - ~ 3 ~ \left( ~ x^2 ~ - ~ 2 ~ x ~ + 2 ~ \right) ~ e^x ~ + ~ 6 \\ \\
	&= ~~ \left( ~ x^3 ~ - ~ 3 ~ \left( ~ x^2 ~ - ~ 2 ~ x ~ + 2 ~ \right) ~ \right) ~ e^x ~ + ~ 6 \\ \\
	&= ~~ \left( ~ x^3 ~ - ~ 3 ~ x^2 ~ + ~ 6 ~ x ~ - 6 ~ \right) ~ e^x ~ + ~ 6 \\ \\
	\end{aligned}$
	
	
	~\\
	
	\newpage
	
	Nach Aufgabe ~ c) ~ ist bereits ein Teilmuster: ~ $I_{n}(x) ~ = ~ ( ~ ... ~ ) ~ e^x ~ + ~ (-1)^{~ ...} ~ n!$ ~ zu vermuten. \\
	


	
	~\\ Pochhammer-Symbol:
	
	~\\
	
	$\begin{aligned}[t]
	(a)_0 ~~ &:= ~~ 1 \\ \\
	(a)_n ~~ &:= ~~ a ~ (a ~ + ~ 1) ~ (a ~ + ~ 2) ~ ... ~ (a ~ + ~ n ~ - ~ 1) \\ \\ \\
	(a)_1 ~~ &~= ~~ a \\ \\
	(a)_2 ~~ &~= ~~ a ~ (a ~ + ~ 1) \\ \\
	(a)_3 ~~ &~= ~~ a ~ (a ~ + ~ 1) ~ (a ~ + ~ 2) \\ \\
	(a)_4 ~~ &~= ~~ a ~ (a ~ + ~ 1) ~ (a ~ + ~ 2) ~ (a ~ + ~ 3) \\ \\
	...
	\end{aligned}$

	~\\
	
	D.h. wir brauchen einen Index (z.B. von einem Summen-, Produkt- oder einem anderen -Zeichen mit Index). Stellt sich die Frage: Wie können wir ~ $I_n(x)$ ~ als Summe schreiben? Probieren wir die Beispiele ~ $I_0, ~ I_1, ~ I_2, ~ I_3$ ~ so zu schreiben, dass sich vielleicht ein Muster zeigt ...
	
	~\\
	
	\[ \left( ~ \sum_{k ~ = ~ 1}^{...} ~ ... ~ \right) ~ e^x ~ + ~ (-1)^{~ ...} ~ n! \]
	
	
	\newpage	

	$\begin{aligned}[t]
	I_0(x) ~~ &= ~~ e^x ~ - ~ 1 \\ \\
	&= ~~ \left( ~ 1 ~ x^0 ~ \right) ~ e^x ~ - ~ 1 \\ \\
	&= ~~ \left( ~ (-0)_0 ~ x^{~ 1 - 1} ~ \right) ~ e^x ~ + ~ \left( -1 \right)^{~ 0 + 1} ~ 0!
	\end{aligned}$
	
	~\\~\\
	
	$\begin{aligned}[t]
	I_1(x) ~~ &= ~~ \left( ~ x ~ - 1 ~ \right) ~ e^x ~ + ~ 1 \\ \\
	&= ~~ \left( ~ x^1 ~ - 1 ~ x^0 ~ \right) ~ e^x ~ + ~ \left( ~ 1 ~ \right) ~ 1 \\ \\
	&= ~~ \left( ~ (-1)_0 ~ x^{~ 2 - 1} ~ + ~ (-1)_1 ~ x^{~ 1 - 1} ~ \right) ~ e^x ~ + ~ \left( -1 \right)^{~ 1 + 1} ~ 1!
	\end{aligned}$
	
	~\\~\\
	
	$\begin{aligned}
	I_2(x) ~~ &= ~~ \left( ~ x^2 ~ - ~ 2 ~ x ~ + 2 ~ \right) ~ e^x ~ - ~ 2 \\ \\
	&= ~~ \left( ~ x^2 ~ - ~ 2 ~ x^1 ~ + 2 ~ x^0 ~ \right) ~ e^x ~ + ~ \left( ~ -1 ~ \right) ~ 2 \\ \\
	&= ~~ \left( ~ (-2)_0 ~ x^{~ 3 - 1} ~ + ~ (-2)_1 ~ x^{~ 2 - 1} ~ + ~ (-2)_2 ~ x^{~ 1 - 1} ~ \right) ~ e^x ~ + ~ \left( -1 \right)^{~ 2 + 1} ~ 2!
	\end{aligned}$
	
	~\\~\\
	
	$\begin{aligned}[t]
	I_3(x) ~~ &= ~~ \left( ~ x^3 ~ - ~ 3 ~ x^2 ~ + ~ 6 ~ x ~ - 6 ~ \right) ~ e^x ~ + ~ \left( ~ 1 ~ \right) ~ 6 \\ \\
	&= ~~ \left( ~ x^3 ~ - ~ 3 ~ x^2 ~ + ~ 6 ~ x^1 ~ - 6 ~ x^0 ~ \right) ~ e^x ~ + ~ \left( ~ 1 ~ \right) ~ 6 \\ \\
	&= ~~ \left( ~ (-3)_0 ~ x^{~ 4 - 1} ~ + ~ (-3)_1 ~ x^{~ 3 - 1} ~ + ~ (-3)_2 ~ x^{~ 2 - 1} ~ + ~ (-3)_3 ~ x^{~ 1 - 1} ~ \right) \\ \\
	& ~~~~ \cdot ~ e^x ~ + ~ \left( -1 \right)^{~ 3 + 1} ~ 3!
	\end{aligned}$
	
	~\\~\\
	
	$\begin{aligned}[t]
	I_4(x) ~~ &= ~~ x^4 ~ e^{x} ~ - ~ 4 ~ I_{~ 4 - 1 ~}(x) \\ \\
	&= ~~ x^4 ~ e^{x} ~ - ~ 4 ~ I_{3}(x) \\ \\
	&= ~~ x^4 ~ e^{x} ~ - ~ 4 ~ \left( ~ x^3 ~ - ~ 3 ~ x^2 ~ + ~ 6 ~ x ~ - 6 ~ \right) \\ \\
	&= ~~ x^4 ~ e^{x} ~ - ~ 4 ~ \left( ~ \left( ~ x^3 ~ - ~ 3 ~ x^2 ~ + ~ 6 ~ x ~ - 6 ~ \right) ~ e^x ~ - ~ 6 ~ \right) \\ \\
	&= ~~ x^4 ~ e^{x} ~ - ~ 4 ~ \left( ~ x^3 ~ - ~ 3 ~ x^2 ~ + ~ 6 ~ x ~ - 6 ~ \right) ~ e^x ~ + ~ 24 \\ \\
	&= ~~ \left( ~ x^4 ~ - ~ 4 ~ \left( ~ x^3 ~ - ~ 3 ~ x^2 ~ + ~ 6 ~ x ~ - 6 ~ \right) ~ \right) ~ e^x ~ + ~ 24 \\ \\
	&= ~~ \left( ~ x^4 ~ - ~ 4 ~ x^3 ~- ~ 12 ~ x^2 ~ - ~ 24 ~ x ~ + 24 ~ \right) ~ e^x ~ + ~ 24 \\ \\
	&= ~~ \left( ~ x^4 ~ - ~ 4 ~ x^3 ~- ~ 12 ~ x^2 ~ - ~ 24 ~ x^1 ~ + 24 ~ x^0 ~ \right) ~ e^x ~ + ~ 24
	\end{aligned}$
	
	~\\
	
	\newpage
	
	Behauptung: \\
	
	\[ \forall n ~ \in ~ \mathbb{N}: ~~ I_{~ (n ~ - ~ 1) ~} (x) ~~ = ~~ e^x ~ \sum_{k ~ = ~ 1}^{n} ~ \left( ~ -n ~ + ~ 1 ~ \right)_{(n ~ - ~ k)} ~ x^{~ k ~ - ~ 1} ~ + ~ (-1)^n ~ (n ~ - ~ 1)! \]
	
	~\\
	~\\
	
	Beweis durch vollständige Induktion.
	
	~\\
	
	\begin{tabularx}{\textwidth}{rX}
		
		Hinweis:  & Ich werde den Beweis ohne Voraussetzungen führen. Denn, wer ihn oder die Eigenschaften des \textit{Pochhammer-Symbols} kennt, wird vielleicht \textit{tricksen}. \\
		
	\end{tabularx}
	
	~\\
	~\\
	
	I.A.:
	
	~\\
	
$\begin{aligned}[t]
	n ~ = ~ 1 ~ : \qquad &\textcolor{white}{=} ~~ I_{~ (1 ~ - ~ 1) ~} (x) \\ \\
	&= ~~ I_0(x) \\ \\
	&= ~~ e^x ~ - ~ 1 \\ \\
	&\oset[1.5ex]{?}{=} ~~ e^x ~ \sum_{k ~ = ~ 1}^{1} ~ \left( ~ -1 ~ + ~ 1 ~ \right)_{(1 ~ - ~ k)} ~ x^{~ k ~ - ~ 1} ~ + ~ (-1)^1 ~ (1 ~ - ~ 1)! \\ \\
	&= ~~ e^x ~ \left( ~ 0 ~ \right)_{(0)} ~ x^{~ 0} ~ - ~ 0! \\ \\
	&= ~~ e^x ~ - ~ 1 \\ \\
	&\square
\end{aligned}$

	\newpage
	
$\begin{aligned}[t]
n ~ = ~ 2 ~ : \qquad &\textcolor{white}{=} ~~ I_{~ (2 ~ - ~ 1) ~} (x) \\ \\
&= ~~ I_1(x) \\ \\
&= ~~ \left( ~ - 1 ~ + ~ x ~ \right) ~ e^x ~ + ~ 1 \\ \\
&\oset[1.5ex]{?}{=} ~~ e^x ~ \sum_{k ~ = ~ 1}^{2} ~ \left( ~ -2 ~ + ~ 1 ~ \right)_{(2 ~ - ~ k)} ~ x^{~ k ~ - ~ 1} ~ + ~ (-1)^2 ~ (2 ~ - ~ 1)! \\ \\
&= ~~ e^x ~ \sum_{k ~ = ~ 1}^{2} ~ \left( ~ -1 ~ \right)_{(2 ~ - ~ k)} ~ x^{~ k ~ - ~ 1} ~ + ~ 1! \\ \\
&= ~~ e^x ~ \left( ~ \left( ~ -1 ~ \right)_{(2 ~ - ~ 1)} ~ x^{~ 1 ~ - ~ 1} ~ + ~ \left( ~ -1 ~ \right)_{(2 ~ - ~ 2)} ~ x^{~ 2 ~ - ~ 1} ~ \right) ~ + ~ 1  \\ \\
&= ~~ e^x ~ \left( ~ \left( ~ -1 ~ \right)_{(1)} ~ x^{~ 0} ~ + ~ \left( ~ -1 ~ \right)_{(0)} ~ x^{~ 1} ~ \right) ~ + ~ 1  \\ \\
&= ~~ e^x ~ \left( ~ -1 ~ + ~ x ~ \right) ~ + ~ 1  \\ \\
&\square
\end{aligned}$

~\\
~\\

I.H.:

\[ \exists n ~ \in ~ \mathbb{N}: ~~ I_{~ (n ~ - ~ 1) ~} (x) ~~ = ~~ e^x ~ \sum_{k ~ = ~ 1}^{n} ~ \left( ~ -n ~ + ~ 1 ~ \right)_{(n ~ - ~ k)} ~ x^{~ k ~ - ~ 1} ~ + ~ (-1)^n ~ (n ~ - ~ 1)! \]
		
	\newpage
	
	I.S.: \setcounter{tc}{0}
	
\begin{flalign*}
%
% \im & & <left_expression> &= <right_expression>
%
& & I_{~ ( ~ (n ~ + ~ 1) ~ - ~ 1 ~ ) ~} (x) ~~ &= ~~ e^x ~ \sum_{k ~ = ~ 1}^{n ~ + ~ 1} ~ \left( ~ -(n ~ + ~ 1) ~ + ~ 1 ~ \right)_{( ~ (n ~ + ~ 1) ~ - ~ k ~ )} ~ x^{~ k ~ - ~ 1} \\ \\
& & & ~~~ ~ + ~ (-1)^{~ (n ~ + ~ 1)} ~ \left( ~ (n ~ + ~ 1) ~ - ~ 1 ~ \right)! \\ \\ \\
%
\im \quad & & I_n (x) ~~ &= ~~ e^x ~ \sum_{k ~ = ~ 1}^{n ~ + ~ 1} ~ \left( ~ -n ~ \right)_{( ~ n ~ + ~ 1 ~ - ~ k ~ )} ~ x^{~ k ~ - ~ 1} \\ \\
& & & ~~~ ~ + ~ (-1)^{~ (n ~ + ~ 1)} ~ n! \\ \\ \\
%
\im \quad & & x^n ~ e^x ~ - ~ n ~ I_{~ (n ~ - ~ 1) ~} (x) ~~ &= ~~ e^x ~ \sum_{k ~ = ~ 1}^{n ~ + ~ 1} ~ \left( ~ -n ~ \right)_{( ~ n ~ + ~ 1 ~ - ~ k ~ )} ~ x^{~ k ~ - ~ 1} \\ \\
& & & ~~~ ~ + ~ (-1)^{~ (n ~ + ~ 1)} ~ n! \\ \\ \\
%
\im \quad & & 0 ~~ &= ~~ e^x ~ \sum_{k ~ = ~ 1}^{n ~ + ~ 1} ~ \left( ~ -n ~ \right)_{( ~ n ~ + ~ 1 ~ - ~ k ~ )} ~ x^{~ k ~ - ~ 1} \\ \\
& & & ~~~ ~ - ~ x^n ~ e^x ~ + ~ n ~ I_{~ (n ~ - ~ 1) ~} (x) \\ \\
& & & ~~~ ~ + ~ (-1)^{~ (n ~ + ~ 1)} ~ n! \\ \\ \\
%
\end{flalign*}


\begin{flalign*}
%
% \im & & <left_expression> &= <right_expression>
%
\im \quad & & 0 ~~ &= ~~ e^x ~ \left( ~ (~ -(n ~ + ~ 1) ~)_{(~ n ~ + ~ 1 ~ -(n ~ + ~ 1) ~)} ~ + ~ \sum_{k ~ = ~ 1}^{n} ~ \left( ~ -n ~ \right)_{( ~ n ~ + ~ 1 ~ - ~ k ~ )} ~ x^{~ k ~ - ~ 1} ~ \right) \\ \\
& & & ~~~ ~ - ~ x^n ~ e^x ~ + ~ n ~ I_{~ (n ~ - ~ 1) ~} (x) \\ \\
& & & ~~~ ~ + ~ (-1)^{~ (n ~ + ~ 1)} ~ n! \\ \\ \\
%
\im \quad & & 0 ~~ &= ~~ e^x ~ \left( ~ x^n ~ + ~ \sum_{k ~ = ~ 1}^{n} ~ \left( ~ -n ~ \right)_{( ~ n ~ + ~ 1 ~ - ~ k ~ )} ~ x^{~ k ~ - ~ 1} ~ \right) \\ \\
& & & ~~~ ~ - ~ x^n ~ e^x ~ + ~ n ~ I_{~ (n ~ - ~ 1) ~} (x) \\ \\
& & & ~~~ ~ + ~ (-1)^{~ (n ~ + ~ 1)} ~ n! \\ \\ \\
%
\im \quad & & 0 ~~ &= ~~ e^x ~ x^n ~ + ~ e^x ~ \sum_{k ~ = ~ 1}^{n} ~ \left( ~ -n ~ \right)_{( ~ n ~ + ~ 1 ~ - ~ k ~ )} ~ x^{~ k ~ - ~ 1} \\ \\
& & & ~~~ ~ - ~ x^n ~ e^x ~ + ~ n ~ I_{~ (n ~ - ~ 1) ~} (x) \\ \\
& & & ~~~ ~ + ~ (-1)^{~ (n ~ + ~ 1)} ~ n! \\ \\ \\
%
\im \quad & & 0 ~~ &= ~~ e^x ~ \sum_{k ~ = ~ 1}^{n} ~ \left( ~ -n ~ \right)_{( ~ n ~ + ~ 1 ~ - ~ k ~ )} ~ x^{~ k ~ - ~ 1} \\ \\
& & & ~~~ ~ + ~ n ~ I_{~ (n ~ - ~ 1) ~} (x) \\ \\
& & & ~~~ ~ + ~ (-1)^{~ (n ~ + ~ 1)} ~ n!
%
\end{flalign*}


\begin{flalign*}
%
% \im & & <left_expression> &= <right_expression>
%
\im \quad & & 0 ~~ &= ~~ e^x ~ \sum_{k ~ = ~ 1}^{n} ~ \left( ~ -n ~ \right)_{( ~ n ~ + ~ 1 ~ - ~ k ~ )} ~ x^{~ k ~ - ~ 1} \\ \\
& & & ~~~ ~ + ~ n ~ \left( ~ e^x ~ \sum_{k ~ = ~ 1}^{n} ~ \left( ~ -n ~ + ~ 1 ~ \right)_{(n ~ - ~ k)} ~ x^{~ k ~ - ~ 1} ~ + ~ (-1)^n ~ (n ~ - ~ 1)! ~ \right) \\ \\
& & & ~~~ ~ + ~ (-1)^{~ (n ~ + ~ 1)} ~ n! \\ \\ \\
%
\im \quad & & 0 ~~ &= ~~ e^x ~ \sum_{k ~ = ~ 1}^{n} ~ \left( ~ -n ~ \right)_{( ~ n ~ + ~ 1 ~ - ~ k ~ )} ~ x^{~ k ~ - ~ 1} \\ \\
& & & ~~~ ~ + ~ e^x ~ \sum_{k ~ = ~ 1}^{n} ~ n ~ \left( ~ -n ~ + ~ 1 ~ \right)_{(n ~ - ~ k)} ~ x^{~ k ~ - ~ 1} ~ + ~ (-1)^n ~ (n ~ - ~ 1)! ~ n \\ \\
& & & ~~~ ~ + ~ (-1)^{~ (n ~ + ~ 1)} ~ n! \\ \\ \\
%
\im \quad & & 0 ~~ &= ~~ e^x ~ \sum_{k ~ = ~ 1}^{n} ~ \left( ~ -n ~ \right)_{( ~ n ~ + ~ 1 ~ - ~ k ~ )} ~ x^{~ k ~ - ~ 1} \\ \\
& & & ~~~ ~ + ~ e^x ~ \sum_{k ~ = ~ 1}^{n} ~ n ~ \left( ~ -n ~ + ~ 1 ~ \right)_{(n ~ - ~ k)} ~ x^{~ k ~ - ~ 1} ~ + ~ (-1)^n ~ n! \\ \\
& & & ~~~ ~ + ~ (-1)^{~ (n ~ + ~ 1)} ~ n! \\ \\ \\
%
\im \quad & & 0 ~~ &= ~~ e^x ~ \sum_{k ~ = ~ 1}^{n} ~ \left( ~ -n ~ \right)_{( ~ n ~ + ~ 1 ~ - ~ k ~ )} ~ x^{~ k ~ - ~ 1} \\ \\
& & & ~~~ ~ + ~ e^x ~ \sum_{k ~ = ~ 1}^{n} ~ n ~ \left( ~ -n ~ + ~ 1 ~ \right)_{(n ~ - ~ k)} ~ x^{~ k ~ - ~ 1} ~ + ~ (-1) ~ (-1)^{~ (n ~ - ~ 1)} ~ n! \\ \\
& & & ~~~ ~ + ~ (-1)^{~ (n ~ + ~ 1)} ~ n!
%
\end{flalign*}


\begin{flalign*}
%
% \im & & <left_expression> &= <right_expression>
%
\im \quad & & 0 ~~ &= ~~ e^x ~ \sum_{k ~ = ~ 1}^{n} ~ \left( ~ -n ~ \right)_{( ~ n ~ + ~ 1 ~ - ~ k ~ )} ~ x^{~ k ~ - ~ 1} ~ + ~ e^x ~ \sum_{k ~ = ~ 1}^{n} ~ n ~ \left( ~ -n ~ + ~ 1 ~ \right)_{(n ~ - ~ k)} ~ x^{~ k ~ - ~ 1} \\ \\
%
\im \quad & & 0 ~~ &= ~~ e^x ~ \left( ~ \sum_{k ~ = ~ 1}^{n} ~ \left( ~ -n ~ \right)_{( ~ n ~ + ~ 1 ~ - ~ k ~ )} ~ x^{~ k ~ - ~ 1} ~ + ~ \sum_{k ~ = ~ 1}^{n} ~ n ~ \left( ~ -n ~ + ~ 1 ~ \right)_{(n ~ - ~ k)} ~ x^{~ k ~ - ~ 1} ~ \right) \\ \\
%
\im \quad & & 0 ~~ &= ~~ e^x ~ \left( ~ \sum_{k ~ = ~ 1}^{n} ~ \left( ~ -n ~ \right)_{( ~ n ~ + ~ 1 ~ - ~ k ~ )} ~ x^{~ k ~ - ~ 1} ~ - ~ \sum_{k ~ = ~ 1}^{n} ~ \underbrace{(-n) ~ \left( ~ -n ~ + ~ 1 ~ \right)_{(n ~ - ~ k)}}_{~ ? ~} ~ x^{~ k ~ - ~ 1} ~ \right) \\ \\
%
\im \quad & & 0 ~~ &= ~~ e^x ~ \left( ~ \sum_{k ~ = ~ 1}^{n} ~ \left( ~ -n ~ \right)_{( ~ n ~ + ~ 1 ~ - ~ k ~ )} ~ x^{~ k ~ - ~ 1} ~ - ~ \sum_{k ~ = ~ 1}^{n} ~ \left( ~ -n ~ \right)_{(n ~ + ~ 1 ~ - ~ k)} ~ x^{~ k ~ - ~ 1} ~ \right) \\ \\
%
\im \quad & & 0 ~~ &= ~~ 0 \\ \\
%
%\im \quad & & 0 ~~ &= ~~ e^x ~ \left( ~ \sum_{k ~ = ~ 1}^{n} ~ \left( ~ -n ~ \right)_{( ~ n ~ + ~ 1 ~ - ~ k ~ )} ~ x^{~ k ~ - ~ 1} ~ + ~ n ~ \left( ~ -n ~ + ~ 1 ~ \right)_{(n ~ - ~ k)} ~ x^{~ k ~ - ~ 1} ~ \right) \\ \\
%
%\im \quad & & 0 ~~ &= ~~ e^x ~ \left( ~ \sum_{k ~ = ~ 1}^{n} ~ \left( ~ -n ~ \right)_{( ~ n ~ + ~ 1 ~ - ~ k ~ )} ~ x^{~ k ~ - ~ 1} ~ + ~ n ~ \left( ~ -n ~ + ~ 1 ~ \right)_{(n ~ - ~ k)} ~ x^{~ k ~ - ~ 1} ~ \right) \\ \\
%
%\im \quad & & 0 ~~ &= ~~ e^x ~ \left( ~ \sum_{k ~ = ~ 1}^{n} ~ \underbrace{\left( ~ \left( ~ -n ~ \right)_{( ~ n ~ + ~ 1 ~ - ~ k ~ )} ~ + ~ n ~ \left( ~ -n ~ + ~ 1 ~ \right)_{(n ~ - ~ k)} ~ \right)}_{ \uset[1.5ex]{?}{=} ~ 0} ~ x^{~ k ~ - ~ 1} ~ \right) \\ \\
%%
\end{flalign*}

~\\

Bemerkung: \setcounter{tc}{0}

\begin{align*}
	&~ \qquad (-n) ~ \left( ~ -n ~ + ~ 1 ~ \right)_{(n ~ - ~ k)} ~~ = ~~ \left( ~ -n ~ \right)_{( ~ n ~ + ~ 1 ~ - ~ k ~ )} \\ \\
	&\eqv \qquad -n ~ \prod_{i ~ = ~ 0}^{n ~ - ~ k ~ - 1} ~ -n ~ + ~ 1 ~ + ~ i ~~ = ~~ \prod_{i ~ = ~ 0}^{n ~ - ~ k ~ - ~ 1 ~ + ~ 1 ~ = ~ n ~ - ~ k} ~ -n ~ + ~ i
\end{align*}

~\\

$\blacksquare$

~\\
~\\

Übrigens: Das Pochhammersymbol wird 0, sobald der Index vom Argument abhängt, diesen um mindestens 1 übersteigt und die Vorzeichen unterschiedlich sind, ist eine 0 im Produkt. Wenn es das noch nicht gibt, könnte man dazu Pochhammer-Null sagen.

	
	
	\newpage
	
	~\\
	~\\
	
	
	\newpage
	
	%Nach unermüdlichen Lösungsangriffen:
	
	Notiz/Gedanken: Hier zeigt sich wieder einmal die Gefahr der Pünktchen- und Kurzschreibweisen. Vielleicht vom Aufgabensteller gut gemeint, oder frech (?) ist die Beschreibung des Pochhammer-Symbols.
	Zunächst denkt man nicht weiter darüber nach, als sich in etwa Vorzustellen, wie die endlichen Produkte aussehen. Am Ende der Induktion bemerkt man, dass sich der Beweis nicht sauber lösen lässt, wenn man sich nicht eine saubere Definition des Pochhammer-Symbols überlegt und zeigt, dass die Differenz 0 ergibt. Man hat gar nichts von der Kurzschreibweise, da man im letzten Schritt genauso viel Zeit verbraucht darüber nachzudenken, wie wenn man gleich ein Produktzeichen eingeführt hätte! Im allgemeinen lohnen sich natürlich Kurzschreibweisen, nur hier in der Aufgabe eben nicht. Was ja auch Teil der Aufgabe gewesen sein kann.



%	&= ~~ \left( ~ x ~ \left( ~ x^3 ~ - ~ 4 ~ x^2 ~- ~ 12 ~ x ~ - ~ 24 ~ \right) ~ + 24 ~ \right) ~ ... \\ \\
%	&= ~~ \left( ~ x ~ \left( ~ x ~ \left( x^2 ~ - ~ 4 ~ x ~- ~ 12 ~ \right) ~ - ~ 24 ~ \right) ~ + 24 ~ \right) ~ ... \\ \\
%	&= ~~ \left( ~ x ~ \left( ~ x ~ \left( ~ x ~ \left( x ~ - ~ 4 ~ \right) ~ - ~ 12 ~ \right) ~ - ~ 24 ~ \right) ~ + 24 ~ \right) ~ ... \\ \\

% % % % % % % % % % % % % % % % % % % % % % % % % % % % % % % % % % % % % % % % % % %


\newpage


% 4

\section*{Aufgabe 4}
%\addcontentsline{toc}{section}{7}

~\\

\begin{enumerate}[leftmargin=*, labelsep=2em, itemsep=3em, label=\alph*)]


% a)

	\item $\begin{aligned}[t]
	\int_{x_0}^{x_1} ~ dx ~ f(x) ~~ &= ~~ \int_{x_0}^{x_1} ~ dx ~ \frac{1}{y(x)} ~ \dydx \\ \\
	&= ~~ {\begin{cases}
		~ u ~ := ~ y(x) ~
	\end{cases}} : \qquad \int_{x_0}^{x_1} ~ dx ~ \frac{1}{u} ~ \dudx \\ \\
	&= ~~ {\begin{cases}
		~ du ~ = ~  dx ~ \dudx ~
	\end{cases}} : \qquad \int_{x_0}^{x_1} ~ du ~ \frac{1}{u} \\ \\
	&= ~~ {\big[ ln ~ \left| ~ u ~ \right| \big]}_{y(x_0)}^{y(x_1)} \\ \\
	&= ~~ ln ~ \left| ~ y(x_1) ~ \right| ~ - ~ ln ~ \left| ~ y(x_0) ~ \right| \\ \\
	&= ~~ ln ~ \frac{ \left| ~ y(x_1) ~ \right| }{ \left| ~ y(x_0) ~ \right| } \qquad, ~~ \text{da ~} y(x_0) ~ \neq ~ 0 \\ \\
	&= ~~ ln ~ \left| ~ \frac{ y(x_1) }{ y(x_0) } ~ \right| \\ \\
	&= ~~ \begin{cases}
		~ sgn ~~ y(x_0) ~ = ~ sgn ~~ y(x_1) ~
	\end{cases} \Rightarrow \qquad ln ~ \frac{ y(x_1) }{ y(x_0) }  \\ \\
	\end{aligned}$
	
	
% b)
	
	\item 
	
	\setcounter{tc}{0}
	
	$\begin{aligned}[t]
	&\textcolor{white}{\Leftrightarrow} ~~ \int_{x_0}^{x_1} ~ f(x) ~~ = ~~ ln ~ \frac{~ y \left( x_1 \right) ~}{~ y \left( x_0 \right) ~} \\ \\
	&\im ~~ {F\left(x\right)|}_{x_0}^{x_1} ~~ = ~~ ln ~ \frac{~ y \left( x_1 \right) ~}{~ y \left( x_0 \right) ~} \\ \\
	&\im ~~ F(x_1) ~ - ~ F(x_0) ~~ = ~~ ln ~ \frac{~ y \left( x_1 \right) ~}{~ y \left( x_0 \right) ~} \\ \\
	&\im ~~ e^{~ F(x_1) ~ - ~ F(x_0)} ~~ = ~~ e^{ ~ ln ~ \frac{~ y \left( x_1 \right) ~}{~ y \left( x_0 \right) ~} } \\ \\
	&\im ~~ e^{~ F(x_1) ~ - ~ F(x_0)} ~~ = ~~ \frac{~ y \left( x_1 \right) ~}{~ y \left( x_0 \right) ~} \\ \\
	&\im ~~ y \left( x_0 \right) ~ e^{~ F(x_1) ~ - ~ F(x_0)} ~~ = ~~ y \left( x_1 \right) \\ \\
	\end{aligned}$
	
	~\\
	
	\setcounter{tc}{0}
	
	\begin{longtable}[l]{l@{\hspace{3em}}l}
		
		\itc, \itc & Definition: bestimmtes Integral. \\ \\
		\itc, \itc & Anwendung: Exponentialfunktion. \\ \\
		\itc & $y \left( x_0 \right)$ ~ nach links. \\ \\
		\itc & ...?
		
	\end{longtable}
	
	
	\newpage
	

% c)

	\item
	
	$y(x) ~ = ~ 0 \quad : ~~ \ddy ~ 0 ~ = ~ \lambda ~ x^{\alpha} ~ 0 ~~ \Rightarrow ~~ 0 ~ = ~ 0$
	
	~\\
	
	\setcounter{tc}{0}
	
	$\begin{aligned}[t]
		&\textcolor{white}{\Leftrightarrow} ~~ \dydx ~~ = ~~ \lambda ~ x^{\alpha} ~ y(x) \\ \\
		&\im ~~ {\begin{cases}
			~ y(x) ~ \neq ~ 0 ~
		\end{cases}} : \qquad \dydx ~ \frac{1}{y(x)} ~~ = ~~ \lambda ~ x^{\alpha} \\ \\
		&\im ~~ \idx ~ \dydx ~ \frac{1}{y(x)} ~~ = ~~ \lambda ~ \idx ~ x^{\alpha} \\ \\
		&\im ~~ {\begin{cases}
			~ u ~ := ~ y(x) ~
			\end{cases}} : \qquad \idx ~ \dudx ~ \frac{1}{u} ~~ = ~~ \lambda ~ \idx ~ x^{\alpha} \\ \\
		&\im ~~ {\begin{cases}
			~ du ~ = ~ dx ~ \dudx ~
			\end{cases}} : \qquad \idu ~ \frac{1}{u} ~~ = ~~ \lambda ~ \idx ~ x^{\alpha} \\ \\
		&\im ~~ ln ~ \left| ~ u ~ \right| ~ + ~ C_1 ~~ = ~~ \lambda ~ \idx ~ x^{\alpha} \qquad, ~~ C_1 ~ \in ~ \mathbb{K} \\ \\
		&\im ~~ ln ~ \left| ~ y(x) ~ \right| ~ + ~ C_1 ~~ = ~~ \lambda ~ \idx ~ x^{\alpha} \\ \\
		&\im ~~ ln ~ \left| ~ y(x) ~ \right| ~ + ~ C_1 ~~ = ~~ \lambda ~ \left( \frac{x^{\alpha ~ + ~ 1}}{\alpha ~ + ~ 1} ~ + ~ C_2 \right) \qquad, ~~ C_2 ~ \in ~ \mathbb{K} ~~, ~~ \alpha ~ \neq ~ -1 \\ \\
		&\im ~~ ln ~ \left| ~ y(x) ~ \right| ~~ = ~~ \lambda ~ \frac{x^{\alpha ~ + ~ 1}}{\alpha ~ + ~ 1} ~ + ~ \lambda ~ C_2 ~ - ~ C_1 \\ \\
		&\im ~~ {\begin{cases}
			~ C_{(1)}(\lambda) ~ := ~ \lambda ~ C_2 ~ - ~ C_1 ~~ ~ \in ~ \mathbb{K} ~
		\end{cases}} : \qquad ln ~ \left| ~ y(x) ~ \right| ~~ = ~~ \lambda ~ \frac{x^{\alpha ~ + ~ 1}}{\alpha ~ + ~ 1} ~ + ~ C_{(1)}(\lambda) \\ \\
		&\im ~~ e^{~ ln ~ \left| ~ y(x) ~ \right|} ~~ = ~~ e^{\lambda ~ \frac{x^{\alpha ~ + ~ 1}}{\alpha ~ + ~ 1} ~ + ~ C_{(1)}(\lambda) } \\ \\
		&\im ~~ \left| ~ y(x) ~ \right| ~~ = ~~ e^{~ \lambda ~ \frac{x^{\alpha ~ + ~ 1}}{\alpha ~ + ~ 1} ~ + ~ C_{(1)}(\lambda) }
	\end{aligned}$
	
	
	\newpage
	
	
	$\begin{aligned}
		&\im ~~ y(x) ~~ = ~~ \pm ~ e^{~ \lambda ~ \frac{x^{\alpha ~ + ~ 1}}{\alpha ~ + ~ 1} ~ + ~ C_{(1)}(\lambda) } \\ \\
		&\im ~~ y(x) ~~ = ~~ \pm ~ e^{~ \lambda ~ \frac{x^{\alpha ~ + ~ 1}}{\alpha ~ + ~ 1}} ~ e^{C_{(1)}(\lambda) } \\ \\
		&\im ~~ {\begin{cases}
			~ C_{(2)}(\lambda) ~ := ~ e^{ C_{(1)}(\lambda) } ~~ ~ \in ~ \mathbb{K} ~
			\end{cases}} : \qquad y(x) ~~ = ~~ \pm ~ e^{~ \lambda ~ \frac{x^{\alpha ~ + ~ 1}}{\alpha ~ + ~ 1}} ~ C_{(2)}(\lambda) \\ \\
		&\im ~~ y(x) ~~ = ~~ \pm ~ C_{(2)}(\lambda) ~ e^{~ \lambda ~ \frac{x^{\alpha ~ + ~ 1}}{\alpha ~ + ~ 1} } \\ \\ \\ % \\
		&\im ~~ {\begin{cases}
			~ \alpha ~ = ~ -1 ~
		\end{cases}} : \qquad ln ~ \left| ~ y(x) ~ \right| ~ + ~ C_1 ~~ = ~~ \lambda ~ \idx ~ \frac{1}{x} \\ \\
		&\im ~~ {\begin{cases}
		~ x ~ > ~ 0 ~
		\end{cases}} : \qquad ln ~ \left| ~ y(x) ~ \right| ~ + ~ C_1 ~~ = ~~ \lambda ~ \left( ~ ln ~ x ~ + ~ C_3 ~ \right) \qquad, ~~ C_3 ~ \in ~ \mathbb{K} \\ \\
		&\im ~~ ln ~ \left| ~ y(x) ~ \right| ~ + ~ C_1 ~~ = ~~ \lambda ~  ~ ln ~ x ~ + ~ \lambda ~ C_3 \\ \\
		&\im ~~ ln ~ \left| ~ y(x) ~ \right| ~~ = ~~ \lambda ~  ~ ln ~ x ~ + ~ \lambda ~ C_3 ~ - ~ C_1 \\ \\
		&\im ~~ {\begin{cases}
			~ C_{(3)}(\lambda) ~ := ~ \lambda ~ C_3 ~ - ~ C_1 ~~ ~ \in ~ \mathbb{K} ~
		\end{cases}} : \qquad ln ~ \left| ~ y(x) ~ \right| ~~ = ~~ \lambda ~  ~ ln ~ x ~ + ~ C_{(3)}(\lambda) \\ \\
		&\im ~~ e^{~ ln ~ \left| ~ y(x) ~ \right|} ~~ = ~~ e^{~ \lambda ~  ~ ln ~ x ~ + ~ C_{(3)}(\lambda)} \\ \\
		&\im ~~ \left| ~ y(x) ~ \right| ~~ = ~~ e^{~ \lambda ~  ~ ln ~ x ~ + ~ C_{(3)}(\lambda)} \\ \\
		&\im ~~ y(x) ~~ = ~~ \pm ~ e^{~ \lambda ~  ~ ln ~ x ~ + ~ C_{(3)}(\lambda)} \\ \\
		&\im ~~ y(x) ~~ = ~~ \pm ~ e^{~ \lambda ~  ~ ln ~ x } ~ e^{~ C_{(3)}(\lambda) } \\ \\
		&\im ~~ y(x) ~~ = ~~ \pm ~ x^{~ \lambda } ~ e^{~ C_{(3)}(\lambda) } \\ \\
		&\im ~~ {\begin{cases}
			~ C_{(4)}(\lambda) ~ := ~ e^{ ~ C_{(3)}(\lambda) } ~~ ~ \in ~ \mathbb{K} ~
		\end{cases}} : \qquad y(x) ~~ = ~~ \pm ~ x^{~ \lambda } ~ C_{(4)}(\lambda) \\ \\
	\end{aligned}$
	
	\newpage
	
	$\begin{aligned}
		&\im ~~ y(x) ~~ = ~~ \pm ~ C_{(4)}(\lambda) ~ x^{~ \lambda }
	\end{aligned}$
	
	~\\	
	~\\
	
	$\begin{aligned}
		&\Rightarrow \qquad \dydx ~ = ~ \lambda ~ x^{\alpha} ~ y(x) \\ \\
		&\Rightarrow ~~ {\begin{cases}
			~ y \left( x \right) ~ \neq ~ 0 ~ : & y \left( x \right) ~ = {\begin{cases}
				~ \alpha ~ \neq ~ 0 ~ : & \pm ~ C ~ e^{~ \lambda ~ \frac{x^{\alpha ~ + ~ 1}}{\alpha ~ + ~ 1} } \\ \\
				~ \alpha ~ = ~ 0 ~ : & \pm ~ C ~ x^{~ \lambda }
				\end{cases}} \\ \\
			~ y \left( x \right) ~ = ~ 0
		\end{cases}} \qquad, ~~ C ~ \in ~ \mathbb{K}
	\end{aligned}$
	
	~\\
	~\\
	
	Hinweis: Konstanten in Zukunft ohne Abhängigkeiten schreiben (unsauber, schneller, eigtl. will man ja auch nicht mehr, als auszudrücken, dass da eine Körperkonstante gewählt werden darf).
	TODO: Gilt die Rückrichtung?
	
	(TODO: Proben)
	
	
	~\\
	~\\
	
	
\setcounter{tc}{0}

\begin{longtable}{l@{\hspace{3em}}l}
	
	\itc & Trennung der Variablen ~ $y(x)$ ~ und ~ $x$. \\ \\
	\itc & ...
	
\end{longtable}

~\\	
~\\ Ich lasse hier ~ $\dydx$ ~ so stehen und schreibe nicht ~ $\dydx ~ = ~ \ddx ~ y(x)$ ~ aus. \\ Diese Kurzschreibweise lohnt sich. Gibt es Fälle, wo es nicht so ist?

~\\	Um die Merkregel der Integral-Substitution gleich sehen zu können, ist es besser, beim Trennen der Symbole ~ $y$ ~ und ~ $x$ ~ den Ausdruck ~ $\dydx$ ~ neben ~ $dx$ ~ des Integrals für die später folgende Integration zu schreiben. Daher: Variable, hier ~ $\frac{1}{y(x)}$ ~ gleich von rechts dran multiplizieren. Stimmt das immer?

~\\
	
	
	
	
	\newpage
	
	
	
	
	% d)
	
	\item

	$y(x) ~ = ~ 0 \quad : ~~ \ddy ~ 0 ~ = ~ exp(\alpha ~ x) ~ 0 ~~ \Rightarrow ~~ 0 ~ = ~ 0$
	
	~\\

	\setcounter{tc}{0}

	$\begin{aligned}[t]
		&\textcolor{white}{\Leftrightarrow} ~~ \dydx ~~ = ~~ exp(\alpha ~ x) ~ y(x) \\ \\
		&\im ~~ {\begin{cases}
			~ y(x) ~ \neq ~ 0 ~
			\end{cases}} : \qquad ~ \dydx ~ \frac{1}{y(x)} ~~ = ~~ e^{~ \alpha ~ x} \\ \\
		&\im ~~ \idx ~ \dydx ~ \frac{1}{y(x)} ~~ = ~~ \idx ~ e^{~ \alpha ~ x} \\ \\
		&\im ~~ {\begin{cases}
			~ u ~ := ~ y(x) ~
			\end{cases}} : \qquad \idx ~ \dudx ~ \frac{1}{u} ~~ = ~~ \idx ~ e^{~ \alpha ~ x} \\ \\
		&\im ~~ {\begin{cases}
			~ du ~ = ~ dx ~ \dudx ~
			\end{cases}} : \qquad \idu ~ \frac{1}{u} ~~ = ~~ \idx ~ e^{~ \alpha ~ x} \\ \\
		&\im ~~ ln \left| ~ u ~ \right| ~ + ~ C_1 ~~ = ~~ \idx ~ e^{~ \alpha ~ x} \qquad, ~~ C_1 ~ \in ~ \mathbb{K} \\ \\
		&\im ~~ { \begin{cases}
			~ \alpha ~ \neq ~ 0 ~
			\end{cases} } : \qquad ln \left| ~ y(x) ~ \right| ~ + ~ C_1 ~~ = ~~ \frac{ e^{~ \alpha ~ x} }{ \alpha } ~ + ~ \frac{C_2}{\alpha} \qquad, ~~ C_2 ~ \in ~ \mathbb{K} \\ \\
		&\im ~~ ln \left| ~ y(x) ~ \right| ~~ = ~~ \frac{ e^{~ \alpha ~ x} }{ \alpha } ~ + ~ \frac{C_2}{\alpha} ~ - ~ C_1 \\ \\
		&\im ~~ {\begin{cases}
			~ C_{(1)} \left( \alpha \right) ~ := ~ C_2 ~ - ~ C_1 ~~ \in ~ \mathbb{K} ~
			\end{cases}} : \qquad ln \left| ~ y(x) ~ \right| ~~ = ~~ \frac{ e^{~ \alpha ~ x} }{ \alpha } ~ + ~ C_{(1)} \left( \alpha \right) \\ \\
		&\im ~~ e^{~ ln \left| ~ y(x) ~ \right|} ~~ = ~~ e^{~ \frac{ e^{~ \alpha ~ x} }{ \alpha } ~ + ~  C_{(1)} \left( \alpha \right)} \\ \\
		&\im ~~ \left| ~ y(x) ~ \right| ~~ = ~~ e^{~ \frac{ e^{~ \alpha ~ x} }{ \alpha } ~ + ~  C_{(1)} \left( \alpha \right)} \\ \\
		&\im ~~ y(x) ~~ = ~~ \pm ~ e^{~ \frac{ e^{~ \alpha ~ x} }{ \alpha } ~ + ~  C_{(1)} \left( \alpha \right)}
	\end{aligned}$
	
	\newpage
	
	$\begin{aligned}[t]
		&\im ~~ y(x) ~~ = ~~ \pm ~ e^{~ \frac{ e^{~ \alpha ~ x} }{ \alpha } } ~ e^{ ~ C_{(1)} \left( \alpha \right) } \\ \\
		&\im ~~ {\begin{cases}
			~ C_{(2)} \left( \alpha \right) ~ := ~ e^{ C_{(1)} \left( \alpha \right) } ~~ \in ~ \mathbb{K} ~
		\end{cases}} : \qquad y(x) ~~ = ~~ \pm ~ e^{~ \frac{ e^{~ \alpha ~ x} }{ \alpha } } ~ C_{(2)} \left( \alpha \right) \\ \\
		&\im ~~ y(x) ~~ = ~~ \pm ~  C_{(2)} \left( \alpha \right) ~ e^{~ \frac{ e^{~ \alpha ~ x} }{ \alpha } } \\ \\ \\ % Anderer Fall
		&\im ~~ {\begin{cases}
			~ \alpha ~ = ~ 0 ~
		\end{cases}} : \qquad ln ~ \left| ~ y(x) ~ \right| ~~ = ~~ \idx ~ 1 \\ \\
		&\im ~~ ln ~ \left| ~ y(x) ~ \right| ~ + ~ C_1 ~~ = ~~ x ~ + ~ C_3 \qquad, ~~ C_3 ~ \in ~ \mathbb{K} \\ \\
		&\im ~~ ln ~ \left| ~ y(x) ~ \right| ~~ = ~~ x ~ + ~ C_3 ~ - ~ C_1 \\ \\
		&\im ~~ {\begin{cases}
			~ C_4 ~ := ~ C_3 ~ - ~ C_1 ~~ \in ~ \mathbb{K} ~
		\end{cases}} : \qquad ln ~ \left| ~ y(x) ~ \right| ~~ = ~~ x ~ + ~ C_4 \\ \\
		&\im ~~ e^{~ ln ~ \left| ~ y(x) ~ \right|} ~~ = ~~ e^{~ x ~ + ~ C_4} \\ \\
		&\im ~~ \left| ~ y(x) ~ \right| ~~ = ~~ e^{~ x ~ + ~ C_4} \\ \\
		&\im ~~ y(x) ~~ = ~~ \pm ~ e^{~ x ~ + ~ C_4} \\ \\
		&\im ~~ y(x) ~~ = ~~ \pm ~ e^{~ x} ~ e^{~ C_4} \\ \\
		&\im ~~ {\begin{cases}
			~ C_5 ~ := ~ e^{C_4} ~~ \in ~ \mathbb{K} ~
		\end{cases}} : \qquad y(x) ~~ = ~~ \pm ~ e^{~ x} ~ C_5 \\ \\
		&\im ~~ y(x) ~~ = ~~ \pm ~ C_5 ~ e^{~ x} \\ \\
	\end{aligned}$
	
	\newpage
	
	
	~\\	
	~\\
	
	$\begin{aligned}
	&\Rightarrow \qquad \dydx ~ = ~ e^{~ \alpha ~ x} ~ y(x) \\ \\
	&\Rightarrow ~~ {\begin{cases}
		~ y \left( x \right) ~ \neq ~ 0 ~ : & y \left( x \right) ~ = {\begin{cases}
			~ \alpha ~ \neq ~ 0 ~ : & \pm ~ C ~ e^{~ \frac{ e^{~ \alpha ~ x} }{ \alpha } } \\ \\
			~ \alpha ~ = ~ 0 ~ : & \pm ~ C ~ e^{~ x}
			\end{cases}} \\ \\
		~ y \left( x \right) ~ = ~ 0
		\end{cases}} \qquad, ~~ C ~ \in ~ \mathbb{K}
	\end{aligned}$
	
	\newpage
	
	\setcounter{tc}{0}
	
	\begin{longtable}{l@{\hspace{3em}}l}
		
		\itc & Trennung der Variablen ~ $y(x)$ ~ und ~ $x$. \\ \\
		\itc & ... \\
		
	\end{longtable}
	
	~\\
	~\\
	
	(TODO) Probe: ~ $y(x) ~ = ~ \pm ~ ...$ ~ :
	
	~\\
	
	$\begin{aligned}[t]
		\ddx ~ y(x) ~~ ...
	\end{aligned}$
	
	~\\
	
	Notiz: Es macht Sinn, Integrationskonstanten immer so lange wie möglich zu behalten? Ein Teil der Aufgabe(n) hat wohl eindringlich darauf abgezielt, daran zu erinnern, dass beim Integrieren auf Sonderfälle von gegebenen Parametern geachtet werden muss!
	
	
	
	

\end{enumerate}






