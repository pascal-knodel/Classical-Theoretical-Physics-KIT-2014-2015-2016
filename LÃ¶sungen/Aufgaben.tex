% !TeX encoding = UTF-8
\documentclass[
% pointednumbers
]{scrreprt}


\usepackage[ngerman]{babel}
\usepackage[utf8]{inputenc}
\usepackage[T1]{fontenc}
\usepackage[fleqn]{amsmath}
\usepackage{amssymb}
\usepackage{amsthm}
\usepackage{longtable}
\usepackage{enumerate}
\usepackage{enumitem}
\usepackage[explicit]{titlesec}
\usepackage{titletoc}
\usepackage{tocloft}
\usepackage{tabularx}
\usepackage{color}
%\usepackage{xcolor}
\usepackage{hyperref}
\usepackage{mathtools}
%\usepackage{empheq}
\usepackage[skins,theorems]{tcolorbox}
\tcbset{highlight math style={enhanced,
		colframe=black,colback=white,arc=0pt,boxrule=1pt}}
\usepackage{cancel}

%\mathindent=0pt

\makeatletter
\newcommand{\oset}[3][0ex]{%
	\mathrel{\mathop{#3}\limits^{
			\vbox to#1{\kern-2\ex@
				\hbox{$\scriptstyle#2$}\vss}}}}
\makeatother

\makeatletter
\newcommand{\uset}[3][0ex]{%
	\mathrel{\mathop{#3}\limits^{
			\vbox to#1{\kern+14\ex@
				\hbox{$\scriptstyle#2$}\vss}}}}
\makeatother



\parindent0pt 

%\newcommand*\Chaptername{Übungsblatt}
%\newcommand*\Sectionname{Aufgabe}

%\renewcommand{\cftchapaftersnum}{~}
%\renewcommand\cftchapafterpnum{\vskip10pt}

%\renewcommand{\cftsecaftersnum}{.~\Sectionname:~}

%\titleformat{\chapter}{\LARGE\bfseries}{\thechapter}{20pt}{\LARGE}
%\titleformat{\section}{\normalfont\large}{\Sectionname~\thesection~}{0pt}{#1}



% Eigene Befehele:

\newcommand{\ddx}{\frac{d}{dx}}
\newcommand{\ddy}{\frac{d}{dy}}
\newcommand{\ddz}{\frac{d}{dz}}
\newcommand{\ddt}{\frac{d}{dt}}
\newcommand{\dfdx}{\frac{df}{dx}}
\newcommand{\dydx}{\frac{dy}{dx}}
\newcommand{\dudx}{\frac{du}{dx}}
\newcommand{\dFdx}{\frac{dF}{dx}}
\newcommand{\dudxdx}{\frac{du}{dx} ~ dx}
\newcommand{\limz}{\underset{x ~ \rightarrow ~ 0}{lim}}
\newcommand{\limi}{\underset{x ~ \rightarrow ~ \infty}{lim}}
%\newcommand{\ie}{\overset{\text{(#1)}}{=}}[1]
%\newcommand{\ei}[1]{\textbf{(#1)}}
\newcommand{\mdq}[1]{``#1"}
\newcommand{\idx}{\int ~ dx}
\newcommand{\idy}{\int ~ dy}
\newcommand{\idu}{\int ~ du}
\newcommand{\idt}{\int ~ dt}
\newcommand{\dd}[1]{\frac{d}{d ~ #1}}

\newcommand{\px}{\frac{\partial}{\partial ~ x}}
\newcommand{\py}{\frac{\partial}{\partial ~ y}}
\newcommand{\pz}{\frac{\partial}{\partial ~ z}}

\def\mathLarge#1{\mbox{\LARGE $#1$}}



\hypersetup{colorlinks,linkcolor=black,urlcolor=blue}
%% or
% \hypersetup{colorlinks=false,pdfborder=000}

% hack into hyperref
\makeatletter
\DeclareUrlCommand\ULurl@@{%
	\def\UrlFont{\ttfamily\color{blue}}%
	\def\UrlLeft{\uline\bgroup}%
	\def\UrlRight{\egroup}}
\def\ULurl@#1{\hyper@linkurl{\ULurl@@{#1}}{#1}}
\DeclareRobustCommand*\ULurl{\hyper@normalise\ULurl@}
\makeatother



% Used to set and reset temporarily
\newcounter{tc}
\newcommand{\itc}{\addtocounter{tc}{1}\textbf{\arabic{tc}}}

\newcounter{tct}
\newcommand{\itct}{\addtocounter{tct}{1}\textbf{\arabic{tct}}}


% Display implication (logical) with number on top, its value is ts's.
\newcommand{\im}{\addtocounter{tc}{1}\oset[1.5ex]{\arabic{tc}}{\Rightarrow}}
% Display equiv (logical) with number on top, its value is ts's.
\newcommand{\eqv}{\addtocounter{tc}{1}\oset[1.5ex]{\arabic{tc}}{\Leftrightarrow}}
% 
\newcommand{\eqs}{\addtocounter{tc}{1}\oset[1.5ex]{\arabic{tc}}{=}}

\newcommand{\qma}[1]{\oset[1.5ex]{?}{#1}}


\newcommand{\imt}{\addtocounter{tct}{1}\oset[1.5ex]{\arabic{tc}.\arabic{tct}}{\Rightarrow}}

\newcommand{\eqst}{\addtocounter{tct}{1}\oset[1.5ex]{\arabic{tc}.\arabic{tct}}{=}}



\begin{document}


% !TeX encoding = UTF-8


\begin{titlepage}
	
	\begin{center}
		
		\textsf{\LARGE Klassische Theoretische Physik I}\\[1.5cm]
		
		\textsc{\Large KIT (WS 2014/2015)}\\[1.25cm]
		
		% Titel
		\newcommand{\HRule}{\rule{\linewidth}{0.25mm}}
		\HRule \\[0.4cm]
		{ \Large \bfseries Mitschrieb zur Vorlesung}
		\HRule \\[1.5cm]
		
		% Autor
		\begin{minipage}{0.5\textwidth}
			\begin{flushleft} \large
				\emph{getippt von} \quad Pascal \textsc{Knodel} 
				
				~\\~\\
				
				\textcolor{red}{pascal.knodel@mail.de}
				
				~\\~\\
				
				Für Fehler oder Nicht-Vollständigkeit übernehme ich keine Verantwortung.
				
			\end{flushleft}
		\end{minipage}
		\hfill
		
		
		\vfill
		
		{\large \today}
		
	\end{center}
	
\end{titlepage}





\tableofcontents

\newpage

Hinweis: \\~\\ Viele der hier ausgeführten Zwischenschritte sind nicht für die Praxis (schriftliche Prüfung) geeignet.

\newpage

\chapter*{Übungsaufgaben}
\addcontentsline{toc}{chapter}{Übungsaufgaben}

% !TeX encoding = UTF-8



%\newpage


\chapter{~}



\newpage



\section{~} % Differentiation

~\\

\begin{enumerate}[leftmargin=*, labelsep=3em, itemsep=3em, label=\alph*)]
	
	\item $\begin{aligned}[t]
	 f'(x) ~~ &= ~~ \ddx ~ x^{\alpha} ~ sin ~ x \\ \\
	 &= ~~ \left( \ddx ~ x^{\alpha} \right) ~ sin ~ x ~ + ~ x^{\alpha} ~ \left( \ddx ~ sin ~ x \right) \\ \\
	 &= ~~ \alpha ~ x^{\alpha ~ - ~ 1} ~ sin ~ x ~ + ~ x^{\alpha} ~ cos ~ x
	\end{aligned}$

	\item $\begin{aligned}[t]
	 f'(x) ~~ &= ~~ \ddx ~ sin ~ x^{\alpha} \\ \\
	 &= ~~ \ddx ~ sin \left( x^{\alpha} \right) \\ \\
	 &= ~~ \left( \frac{d}{d ~ \left( x^{\alpha}\right)} ~ sin ~ x^{\alpha} \right) ~ \left( \ddx ~ x^{\alpha} \right) \\ \\
	 &= ~~ cos ~ x^{\alpha} ~ \alpha ~ x^{\alpha ~ - ~ 1}
	\end{aligned}$
	
	\item $\begin{aligned}[t]
	 f'(x) ~~ &= ~~ \ddx ~ sin^{\alpha} ~ x \\ \\
	 &= ~~ \ddx ~ {\left( sin ~ x \right)}^{\alpha} \\ \\
	 &= ~~ \left( \frac{d}{d \left( sin ~ x \right)} ~ {\left( sin ~ x \right)}^{\alpha} \right) ~ \left( \ddx ~ sin ~ x \right) \\ \\
	 &= ~~ \alpha ~ {\left( sin ~ x \right)}^{\alpha ~ - ~ 1} ~ cos ~ x
	\end{aligned}$
	
	\item $\begin{aligned}[t]
	 f'(x) ~~ &= ~~ \ddx ~ x^{\alpha} ~ sin ~ \frac{1}{x} \\ \\
	 &= ~~ \left( \ddx ~ x^{\alpha} \right) ~ sin ~ \frac{1}{x} ~ + ~ x^{\alpha} ~ \left( \ddx ~ sin ~ \frac{1}{x} \right) \\ \\
	 &= ~~ \alpha ~ x^{\alpha ~ - ~ 1} ~ sin ~ \frac{1}{x} ~ + ~ x^{\alpha} ~ \left( \left( \frac{d}{d ~ \left( sin ~ \frac{1}{x} \right)} ~ sin ~ \frac{1}{x} \right) ~ \left( \ddx ~ \frac{1}{x} \right) \right) \\ \\
	 &= ~~ \alpha ~ x^{\alpha ~ - ~ 1} ~ sin ~ \frac{1}{x} ~ + ~ x^{\alpha}  ~ cos ~ \frac{1}{x} ~ \left( \ddx ~ x^{-1} \right) \\ \\
	 &= ~~ \alpha ~ x^{\alpha ~ - ~ 1} ~ sin ~ \frac{1}{x} ~ + ~ x^{\alpha}  ~ cos ~ \frac{1}{x} ~ \left( -1 \right) ~ x^{-2}
	\end{aligned}$
	
\end{enumerate}



\newpage



Bestimmung der Grenzwerte aus Teilaufgabe ~ 1, d) ~ von ~ $f(x)$ ~ und ~ $f'(x) ~ = ~ \dfdx$ :\\

\begin{description}[leftmargin=*, labelsep=3em, itemsep=3em]
	
	\item[\textnormal{d)}] $f(x) ~ = ~ x^{\alpha} ~ sin ~ \frac{1}{x}$ \hfill \break
	
	\[ \limz ~ \left[ x^{\alpha} ~ sin ~ \frac{1}{x} \right] ~~ \neq ~~ \limz ~ \left[ x^{\alpha} \right] ~ \limz ~ \left[ sin ~ \frac{1}{x} \right] ~~ = ~~ 0 ~ \limi ~ \left[ sin ~ x \right] ~~ = ~~ 0 ~ ? ~~ = ~~ ? \]
	
	~\\D.h., wir wissen noch nichts über den Grenzwert.
	
	~\\
	
	\textbf{Möglichkeit:} \qquad Abschätzen und danach einschlie{"s}en.
	
	~\\
	
	$\begin{aligned}
	&\textcolor{white}{\Leftrightarrow} ~~ x^{\alpha} ~ sin ~ \frac{1}{x} ~ \in ~ \big[ ~ x^{\alpha} \cdot [-1, ~ 1] ~ \big] \\ \\
	&\Leftrightarrow ~~ {-x}^{\alpha} ~ \leq ~ f(x) ~ \leq ~ x^{\alpha} \\ \\
	&\Leftrightarrow ~~ \limz ~ \left[ {-x}^{\alpha} \right] ~ \leq ~ f(x) ~ \leq ~ \limz ~ \left[ x^{\alpha} \right] \\ \\
	&\Leftrightarrow ~~ 0 ~ \leq ~ f(x) ~ \leq ~ 0 \\ \\
	&\Leftrightarrow ~~ \limz ~ \left[ x^{\alpha} ~ sin ~ \frac{1}{x} \right] ~ = ~ 0
	\end{aligned}$
	
	\item[\textnormal{d)}] $\dfdx ~ = ~ \limz ~ \left[ \alpha ~ x^{\alpha ~ - ~ 1} ~ sin ~ \frac{1}{x} ~ - ~ x^{\alpha ~ - ~ 2}  ~ cos ~ \frac{1}{x} \right]$
	
	~\\
	
	$\begin{aligned}
	&\textcolor{white}{=} \limz ~ \left[ \alpha ~ x^{\alpha ~ - ~ 1} ~ sin ~ \frac{1}{x} ~ - ~ x^{\alpha ~ - ~ 2}  ~ cos ~ \frac{1}{x} \right] \\ \\
	& ~~ \neq ~~ \limz ~ \left[ \alpha ~ x^{\alpha ~ - ~ 1} ~ sin ~ \frac{1}{x} \right] ~ - ~ \limz ~ \left[ x^{\alpha}  ~ cos ~ \frac{1}{x} ~  x^{-2} \right] \\ \\
	& ~~ = ~~ 0 ~ - ~ \limz ~ \left[ x^{\alpha}  ~ cos ~ \frac{1}{x} ~  x^{-2} \right] \\ \\
	& ~~ = ~~ - ~ \limz ~ \left[ x^{\alpha ~ - ~ 2}  ~ cos ~ \frac{1}{x} \right] \\ \\
	& ~~ = ~~ {\begin{cases}
		~ a > 2: \qquad & 0 \\ \\
		~ a = 2: \qquad & \limz ~ cos ~ \frac{1}{x} ~ = ~ \limi ~ cos ~ x ~ = ~ ? \\ \\
		~ ... 
	\end{cases}}
	\end{aligned}$\\
	
	~\\D.h., wir wissen auch hier noch nichts über den Grenzwert.

	~\\
	
	\textbf{Möglichkeit:} \qquad Abschätzen und danach einschlie{"s}en.
	
	~\\
	
	$\begin{aligned}
	\textcolor{white}{\Leftrightarrow} ~~ x^{\alpha ~ - ~ 1} ~ \left( a ~ sin ~ \frac{1}{x} ~ - ~ \frac{1}{x} ~ cos ~ \frac{1}{x} \right) ~ &\in ~ x^{\alpha ~ - ~ 1} ~ \big[ ~ a ~ \left[ -1, ~ 1 \right] ~ - ~ \frac{1}{x} ~ \left[ -1, ~ 1 \right] ~ \big] \\ \\
	&\in ~~ x^{\alpha ~ - ~ 1} ~ \left[ ~ -a ~ - ~ \frac{1}{x}, ~ a ~ + ~ \frac{1}{x} ~ \right]
	\end{aligned}$ \\
	
	~\\
	
	$\begin{aligned}
	&\Leftrightarrow ~~ x^{\alpha ~ - ~ 1} ~ \left( -\alpha ~ - ~ \frac{1}{x} \right) ~ \leq ~ \dfdx ~ \leq ~ x^{\alpha ~ - ~ 1} ~ \left( \alpha ~ + ~ \frac{1}{x} \right) ~ \\ \\
	&\Leftrightarrow ~~ -\alpha ~ x^{\alpha ~ - ~ 1} ~ - ~ x^{\alpha ~ - ~ 2} ~ \leq ~ \dfdx ~ \leq ~ \alpha ~ x^{\alpha ~ - ~ 1} ~ + ~ x^{\alpha ~ - ~ 2} \\ \\
	&\Leftrightarrow ~~ \limz ~ \left[ -\alpha ~ x^{\alpha ~ - ~ 1} ~ - ~ x^{\alpha ~ - ~ 2} \right] \leq ~ \limz ~ \dfdx ~ \leq ~ \limz ~ \left[ \alpha ~ x^{\alpha ~ - ~ 1} ~ + ~ x^{\alpha ~ - ~ 2} \right] \\ \\
	&\Leftrightarrow ~~ {\begin{cases}
		~ a > 2: \qquad & \limz ~ \dfdx ~ = ~ 0 \\ \\
		~ a = 2: \qquad & -1 ~ \leq \limz ~ \dfdx ~ \leq ~ 1 \\ \\
		~ ... 
	\end{cases}}
	\end{aligned}$ \\
	
	~\\ D.h. wir wissen auch hier noch nichts über den Grenzwert für ~ $\alpha ~ \leq ~ 2$ ~ .
	
	~\\
	~\\
	
	$\mdq{\text{\textit{Besser}}}$ abschätzen und noch Mal einschließen:
	
	~\\
	
	$\begin{aligned}
	&\textcolor{white}{\Leftrightarrow} ~~ -\alpha ~ x^{\alpha ~ - ~ 1} ~ - ~ x^{\alpha ~ - ~ 2} ~ \leq ~ \dfdx ~ \leq ~ \alpha ~ x^{\alpha ~ - ~ 1} ~ + ~ x^{\alpha ~ - ~ 2} \\ \\
	&\Rightarrow ~~ -\alpha ~ x^{\alpha ~ - ~ 1} ~ - ~ x^{\alpha} ~ \leq ~ \dfdx ~ \leq ~ \alpha ~ x^{\alpha ~ - ~ 1} ~ + ~ x^{\alpha} \\ \\
	&\Leftrightarrow ~~ \limz ~ \left[ -\alpha ~ x^{\alpha ~ - ~ 1} ~ - ~ x^{\alpha} \right] \leq ~ \limz ~ \dfdx ~ \leq ~ \limz ~ \left[ \alpha ~ x^{\alpha ~ - ~ 1} ~ + ~ x^{\alpha} \right] \\ \\
	&\Leftrightarrow ~~ 0 \leq ~ \limz ~ \dfdx ~ \leq ~ 0 \\ \\
	&\Leftrightarrow ~~ \limz ~ \dfdx ~ = ~ 0
	\end{aligned}$ \\
	

\end{description}



\section{~}

~\\


\begin{enumerate}[leftmargin=*, labelsep=3em, itemsep=3em]
	
	\item $\begin{aligned}[t]
	F(x) ~~ &= ~~ \idx ~ x^{\alpha} \\ \\
	&= ~~ \frac{1}{\alpha ~ + ~ 1} ~ x^{\alpha ~ + ~ 1} ~ + ~ C \qquad, ~~ C ~ \in ~ \mathbb{K}
	\end{aligned}$
	
	\item $\begin{aligned}[t]
	F(x) ~~ &= ~~ \idx ~ \underbrace{x^2}_{:\int} ~ \underbrace{cos ~ x}_{:\ddx} \\ \\
	&= ~~ x^2 ~ \idx ~ cos ~ x ~ - ~ \idx ~ \left( \ddx ~ x^2 \right) ~ \left( \idx ~ cos ~ x \right) \\ \\
	&= ~~ x^2 ~ sin ~ x ~ - ~ 2 ~ \idx ~ \underbrace{x}_{:\int} ~ \underbrace{sin ~ x}_{:\ddx} \\ \\
	&= ~~ x^2 ~ sin ~ x ~ - ~ 2 ~ \left( x ~ \idx ~ sin ~ x ~ - ~ \idx ~ \left( \ddx ~ x \right) ~ \left( \idx ~ sin ~ x \right) \right) \\ \\
	&= ~~ x^2 ~ sin ~ x ~ - ~ 2 ~ \left( x ~ \left( - cos ~ x \right) ~ - ~ \idx ~ 1 ~ \left( - cos ~ x \right) \right) \\ \\
	&= ~~ x^2 ~ sin ~ x ~ - ~ 2 ~ \left( - x ~ cos ~ x ~ + ~ sin ~ x \right)
	\end{aligned}$

\end{enumerate}
	
	

\newpage
	

\begin{description}[leftmargin=*, labelsep=3em, itemsep=3em]
	
	\item[\textnormal{c) \quad i)}] \hfill
	
	~\\
	
	\setcounter{tc}{0}
	
	$\begin{aligned}[t]
	&\textcolor{white}{\Leftrightarrow} ~~ L(x) ~ + ~ L(y) ~~ = ~~ L(xy) \\ \\
	&\im \qquad \ddx ~ L(x) ~ + ~ \ddx ~ L(y) ~~ = ~~ \ddx ~ L(xy) \\ \\
	&\im \qquad \frac{1}{x} ~ + ~ 0 ~~ = ~~ \left( \dd{(xy)} ~ L(xy) \right) ~ \left( \ddx ~ xy \right) \\ \\
	&\im \qquad \frac{1}{x} ~~ = ~~ \frac{1}{xy} ~ 1 ~ y \\ \\
	&\im \qquad \frac{1}{x} ~~ = ~~ \frac{1}{x} \\ \\
	&\blacksquare
	\end{aligned}$ \\
	
	~\\
	
	Dies zeigt die Äquivalenz. Um etwas direkter von ~ $L(x) ~ + ~ L(y)$ ~ nach ~ $L(xy)$ ~ (oder umgekehrt) zu kommen, hilft die in diesem Beweis verwendete Ableitungsregel in einer Gleichungskette ...
	
	~\\
	~\\
	
	\newpage
	
	\textbf{Herleitung:}
	
	~\\
	
	$\begin{aligned}[t]
	L(x) ~ + ~ L(y) ~~ &= ~~ L(x) ~ + ~ C \qquad, ~~ C ~ = ~ L(y) ~ \in ~ \mathbb{K} \\ \\
	&= ~~ \idx ~ \ddx ~ L(x) \\ \\
	&= ~~ \idx ~ \frac{1}{x} \\ \\
	&= ~~ \idx ~ \frac{1}{x} ~ 1 \\ \\
	&= ~~ \idx ~ \frac{1}{x} ~ \frac{y}{y} \\ \\
	&= ~~ \idx ~ \frac{1}{xy} ~ y \\ \\
	&= ~~ \idx ~ \left( \dd{(xy)} ~ L(xy) \right) ~ \left( \ddx ~ xy \right) \\ \\
	&= ~~ \idx ~ \ddx ~ L(xy) \\ \\
	&= ~~ L(xy) ~ + ~ C \qquad, ~~ C ~ := ~ 0 ~ \in ~ \mathbb{K} \\ \\
	&= ~~ L(xy) \\ \\
	 &\blacksquare
	\end{aligned}$ \\
	
	
	\newpage
	
	
	\item[\textnormal{c) \quad ii)}] \hfill
	
	~\\
	
	$\begin{aligned}[t]
	&\textcolor{white}{\Leftrightarrow} ~~ L(x^{\alpha}) ~~ = ~~ \alpha ~ L(x) \\ \\
	&\Leftrightarrow ~~ \ddx ~ L(x^{\alpha}) ~~ = ~~ \ddx ~ \alpha ~ L(x) \\ \\
	&\Leftrightarrow ~~ \dd{(x^{\alpha})} ~ L(x^{\alpha}) ~ \ddx ~ x^{\alpha} ~~ = ~~ \alpha ~ \ddx ~ L(x) \\ \\
	&\Leftrightarrow ~~ \frac{1}{x^{\alpha}} ~ \alpha ~ x^{\alpha ~ - ~ 1} ~~ = ~~ \alpha ~\frac{1}{x} \\ \\
	&\Leftrightarrow ~~ \alpha ~\frac{1}{x} ~~ = ~~ \alpha ~\frac{1}{x} \\ \\
	&\blacksquare
	\end{aligned}$ \\
	
	~\\
	
	Dies zeigt die Äquivalenz. Um etwas direkter von ~ $L(x^{\alpha})$ ~ nach ~ $\alpha ~ L(x)$ ~ (oder umgekehrt) zu kommen, hilft die in diesem Beweis verwendete Ableitungsregel in einer Gleichungskette ...
	
	~\\
	~\\
	
	$\begin{aligned}[t]
	L(x^{\alpha}) ~~ &= ~~ \idx ~ \ddx ~ L(x^{\alpha}) \\ \\
	&= ~~ \idx ~ \dd{(x^{\alpha})} ~ L(x^{\alpha}) ~ \ddx ~ x^{\alpha} \\ \\
	&= ~~ \idx ~ \frac{1}{x^{\alpha}} ~ \alpha ~ x^{\alpha ~ - ~ 1} \\ \\
	&= ~~ \alpha ~ \idx ~ \frac{1}{x} \\ \\
	&= ~~ \alpha ~ \idx ~ \ddx ~ L(x) \\ \\
	&= ~~ \alpha ~ L(x) ~ + ~ C \qquad, ~~ C ~ := ~ 0 ~ \in ~ \mathbb{K} \\ \\
	&= ~~ a ~ L(x) \\ \\
	&\blacksquare
	\end{aligned}$ \\
	
	
	\newpage
	
	
	\item[\textnormal{c) \quad iii)}] \hfill
	
	~\\
	
	Spielchen: \qquad Finde mindestens zwei Fehler in ...
	
	~\\
	
	\setcounter{tc}{0}
	
	$\begin{aligned}[t]
	&\textcolor{white}{\Leftrightarrow} ~~ \frac{df^{-1}}{dy} ~~ = ~~ \frac{1}{\frac{df}{dy}} \\ \\
	&\im ~~ \frac{df^{-1}}{dy} ~~ = ~~ \frac{1}{\ddy ~ f(x)} \\ \\
	&\im ~~ \frac{df^{-1}}{dy} ~~ = ~~ \frac{1}{\ddy ~ y} \\ \\
	&\im ~~ \ddy ~ f^{-1}(y) ~~ = ~~ 1 \\ \\
	&\im ~~ \ddy ~ x ~~ = ~~ 1 \\ \\
	&\im ~~ 0 ~~ = ~~ 1
	\end{aligned}$ \\
	
	~\\	
	
	\underline{Fehler:} \\
	
	1. Gleichung falsch übernommen. \\
	
	2. In ~ $^{4}$ ~ sind ~ $x$ ~ und ~ $y$ ~ abhängig, ~ $y$ ~ ist ja als Funktion von ~ $x$ ~ definiert, ~ $y ~ = ~ f(x)$ ~ . Die Abhängigkeit hilft in einem richtigen Beweis ...
	
	~\\

	\setcounter{tc}{0}
	
	$\begin{aligned}[t]
	&\textcolor{white}{\Leftrightarrow} ~~ \frac{df^{-1}}{dy} ~~ = ~~ \frac{1}{ \frac{df}{dx} } \\ \\
	&\im \qquad \ddy ~ f^-1(y) ~~ = ~~ \frac{1}{ \ddx ~ f(x) } \\ \\
	&\im \qquad \left( \ddy ~ f^-1(y) \right) ~ \left( \ddx ~ f(x) \right) ~~ = ~~ 1 \\ \\
	&\im \qquad \ddx ~ f^-1\left( ~ f(x) ~ \right) ~~ = ~~ 1 \\ \\
	&\im \qquad \ddx ~ x ~~ = ~~ 1 \\ \\
	&\im \qquad 1 ~~ = ~~ 1 \\ \\
	&\blacksquare
	\end{aligned}$ \\
	
	~\\
	
	Dies zeigt die Äquivalenz. Um etwas direkter von ~ $\frac{df^{-1}}{dy}$ ~ nach ~ $\frac{1}{ \frac{df}{dx} }$ ~ zu kommen, oder umgekehrt, hilft das neutrale Produkt und der neutrale Quotient ~ $1$ ~ , unter der gegebenen Bedingung, dass ~ $\dfdx ~ \neq ~ 0$ ~ ...
	
	
	\newpage
	
	$\begin{aligned}[t]
	\frac{df^{-1}}{dy} ~~ &= ~~ \ddy ~ f^{-1}(y) \\ \\
	&= ~~ \ddy ~ f^{-1}(y) ~ 1 \\ \\
	&= ~~ \left( \ddy ~ f^{-1}(y) \right) ~ \mathLarge{ \frac{ \underset{ \dfdx }{~} }{ \overset{ \dfdx }{~} } } \\ \\
	&= ~~ \frac{ \left( \ddy ~ f^{-1}(y) \right) ~ \dfdx }{ \overset{ \dfdx }{~} } \\ \\
	&= ~~ \frac{ \left( \ddy ~ f^{-1}(y) \right) ~ \left( \ddx ~ f(x) \right) }{ \overset{ \dfdx }{~} } \\ \\
	&= ~~ \frac{ \ddx ~ f^{-1}\left( ~ f(x) ~ \right) }{ \overset{ \dfdx }{~} } \\ \\
	&= ~~ \frac{ \ddx ~ x }{ \overset{ \dfdx }{~} } \\ \\
	&= ~~ \frac{ 1 }{ \overset{ \dfdx }{~} } \\ \\
	&\blacksquare
	\end{aligned}$ \\
	
	
	\newpage
	
	
	\item[\textnormal{c) \quad iv)}] \hfill
	
	~\\
	
	$\begin{aligned}[t]
	\frac{dL^{-1}}{dy} ~~ &= ~~ \frac{1}{ \frac{dL}{dx} } \\ \\
	&= ~~ \frac{1}{ \frac{1}{x} } \\ \\
	&= ~~ x \\ \\
	&= ~~ L^{-1}\left( ~ L(x) ~ \right) \\ \\
	&= ~~ L^{-1}(y) \\ \\
	&\blacksquare
	\end{aligned}$ \\
	
	
	
	
\end{description}

 



% !TeX encoding = UTF-8



\newpage



\chapter{~}


\underline{Fragen zu den Aufgaben oder Allgemeinem:}

~\\

\begin{enumerate}
	
	\item Was sind die Stammfunktionen von ~ $\frac{1}{x}$ ~ ? \\
	
	\begin{tabularx}{0.88\textwidth}{lX}
		$\bullet$ & $\int ~ dy ~ \frac{1}{x} ~~ = ~~ ln ~ \left| x \right| ~ + ~ C \qquad, ~~ C ~ \in ~ \mathbb{K}$
	\end{tabularx}
	
	~\\
	
	\item Welche Beziehung gilt bei genau einer beliebigen stetigen Funktion die keine Nullstellen hat zwischen Vorzeichen von (mindestens zwei) ihrer beliebigen Parametrisierungen? \\
	
	\begin{tabularx}{0.88\textwidth}{lX}
		$\bullet$ & Sei ~ $\text{\underline{sf}}(x)$ ~ die beschriebene stetige Funktion und seien ~ $\text{\underline{sf}}\left(~ p_{~ i} ~\right)$ ~ Parametrisierungen mit ~ $i ~ \in ~ \{ i_1, ~ i_2 \} \subset ~ \mathbb{N}$ ~ . So gilt: \newline\newline ~ $sgn ~~ \text{\underline{sf}}\left(~ p_{~ i_1} ~\right) ~~ = ~~ sgn ~~ \text{\underline{sf}}\left(~ p_{~ i_2} ~\right)$ ~ .
	\end{tabularx}
	
	~\\
	
	\item Was ist zu tun, wenn in der Aufgabe steht: ~ \textquotedblleft~\textit{Drücken Sie ~ $F_1$ ~ durch ~ $F_2$ ~ aus.}~\textquotedblright ~ ? \\
	
	\begin{tabularx}{0.88\textwidth}{lX}
		$\bullet$ & Eine Gleichung ~ $... ~ F_2 ~ ... ~~ = ~~ ...$ ~ in die Form ~ $F_1 ~~ = ~~ ... ~ F_2 ~ ...$ ~ bringen.
	\end{tabularx}
	
%	~\\
%	
%	\item Der Ausdruck ~ $ \int_{~ x_1}^{x_2} ~ dx ~ f(x)$ ~ ist gegeben. Wie sind hier alle Stammfunktionen von der inneren Funktion des unbestimmten Integrals ~ $f(x)$ ~ formalisierbar? \\
%	
%	\begin{tabularx}{0.88\textwidth}{lX}
%		$\bullet$ & $ F(x) ~ + ~ C ~~ := ~~ \int ~dx ~ f(x) \qquad, ~~ C ~ \in ~ \mathbb{K} $ ~ . \newline Bestimmte und unbestimmte Integrale nicht verwechseln!
%	\end{tabularx}
	
	
	\newpage
	
	
	\item Warum ist ~ $ln ~ \left| x \right|$ ~ die Stammfunktion von ~ $\frac{1}{x}$ ~ ? \\
	
	\begin{tabularx}{0.88\textwidth}{lX}
		$\bullet$ & (TODO: Antwort)
	\end{tabularx}
	
	
	\newpage
	
	
	\item Ist dieser Beweis richtig  ... ~\\
	
	\begin{tabularx}{0.88\textwidth}{lX}
		Behauptung: & $F(x)$ ~ ist eine spezielle Lösung der Differentialgleichung ~ $\dydx ~ = ~ f(x) ~ y(x)$ ~ .
	\end{tabularx}
	 
	~\\
	~\\
	
	$ (*) ~ := ~ \begin{cases}
	~ F(x) ~~ &:= ~~ \int_{x_0}^{x_1} ~ dx ~ f(x) \\ \\
	~ y(x) ~~ &:= ~~ F(x)
	\end{cases}$ \\
	
	~\\
	
	$\begin{aligned}[t]
	(*) ~ &\Rightarrow ~~ \int_{x_0}^{x_1} ~ dx ~ \frac{1}{y(x)} ~ \dydx ~~ = ~~ \int_{x_0}^{x_1} ~ dx ~ f(x) \\ \\
	&\Leftrightarrow ~~ \int_{x_0}^{x_1} ~ dx ~ \frac{1}{F(x)} ~ \dFdx ~~ = ~~ F(x) \\ \\
	&\Leftrightarrow ~~ \int_{x_0}^{x_1} ~ dx ~ \frac{1}{F(x)} ~ f(x) ~~ = ~~ F(x) \\ \\
	&\Leftrightarrow ~~ \int_{x_0}^{x_1} ~ dx ~ \frac{1}{F(x)} ~ f(x) ~~ = ~~ \int_{x_0}^{x_1} ~ dx ~ f(x) \\ \\
	&\Leftrightarrow ~~ \int_{x_0}^{x_1} ~ dx ~ \frac{1}{F(x)} ~ f(x) ~~ = ~~ \int_{x_0}^{x_1} ~ dx ~ 1 ~ f(x) \\ \\
	&\Leftrightarrow ~~ \frac{1}{F(x)} ~~ = ~~ 1\\ \\
	&\Leftrightarrow ~~ F(x) ~~ = ~~ 1 \\ \\
	&\Leftrightarrow ~~ y(x) ~~ = ~~ 1
	\end{aligned}$
	
	
	\newpage
	
	
	Probe: ~ $y(x) ~ = ~ 1$ ~ :
	
	~\\
	
	$\begin{aligned}[t]
	&\textcolor{white}{\Leftrightarrow} ~~ \ddx ~ y(x) ~~ = ~~ f(x) ~  y(x) \\ \\
	&\Leftrightarrow ~~ \ddx ~ 1 ~~ = ~~ f(x) ~  1 \\ \\
	&\Leftrightarrow ~~ 0 ~~ = ~~ f(x) \\ \\
	&\Leftrightarrow ~~ wahr
	\end{aligned}$
	
	~\\
	~\\
	
	...?
	
	~\\
	
	\begin{tabularx}{0.88\textwidth}{lX}
		$\bullet$ & (TODO: Antwort)
	\end{tabularx}
	
	~\\

%	\item ...
	
	
\end{enumerate}





\newpage




\section{~}

~\\

\begin{enumerate}[leftmargin=*, labelsep=2em, itemsep=3em, label=\alph*)]

% a)

	\item \hfill \break
	
	\begin{longtable}[l]{rl}
		
		Forderung:  &  Anwendung von partieller Integration. \\
		
		\\
		
		  Bekannt:  &  $\ddy ~ e^y ~ = ~ e^y$.
		
	\end{longtable}
	
	~\\ $\begin{aligned}[t]
		I_0(x) ~~ &= ~~ \int_{0}^{x} ~ dy ~ y^0 ~ e^y \\ \\
		&= ~~ \int_{0}^{x} ~ dy ~ 1 ~ e^y \\ \\
		&= ~~ \int_{0}^{x} ~ dy ~ \underbrace{1}_{ \int } ~ \underbrace{ e^y }_{ \ddy } \\ \\
		&= ~~ \left[ ~ 1 ~ \left( \ddy ~ e^y \right) ~ \right]_{0}^{x} ~ - ~ \int_{0}^{x} ~ dy ~ \left( \ddy ~ 1 \right) ~ e^y \\ \\
		&= ~~ \left[ ~ e^y ~ \right]_{0}^{x} ~ - ~ \int_{0}^{x} ~ dy ~ 0 ~ e^y \\ \\
		&= ~~ e^x ~ - ~ e^0 ~ - ~ \int_{0}^{x} ~ dy ~ 0 \\ \\
		&= ~~ e^x ~ - ~ 1 ~ - ~ 0 \\ \\
		&= ~~ e^x ~ - ~ 1 \\ \\
	\end{aligned}$
	
	\newpage
	

% b)
	
	\item \hfill \break
	
	$I_n(x) ~ = ~ \int_{0}^{x} ~ dy ~ y^n ~ e^y ~~ \xLeftrightarrow{\text{~ Partielle Integration ~}} ~~ \int_{0}^{x} ~ dy ~ y^{ ~ n ~ - ~ 1} ~ e^y ~ = ~ I_{~ (n ~ - ~ 1)}(x)$
	
	~\\
	
	$\begin{aligned}[t]
	I_n(x) ~~ &= ~~ \int_{0}^{x} ~ dy ~ \underbrace{ y^n }_{ \int } ~ \underbrace{ e^y }_{ \ddy } \\ \\
	&= ~~ \left[ y^n ~ \left( \ddy ~ e^y \right) \right]_{ 0 }^{ x } ~ - ~ \int_{0}^{x} ~ dy ~ \left( \ddy ~ y^n \right) ~ e^y \\ \\
	&= ~~ \left[ ~ y^n ~ e^y ~ \right]_{ 0 }^{ x } ~ - ~ \int_{0}^{x} ~ dy ~ n ~ y^{ ~ n ~ - ~ 1} ~ e^y \\ \\
	&= ~~ x^n ~ e^x ~ - ~ 0 ~ ... ~ - ~ n ~ \underbrace{ \int_{0}^{x} ~ dy ~ y^{ ~ n ~ - ~ 1} ~ e^y }_{ I_{~ (n ~ - ~ 1) ~}(x) } \\ \\
	&= ~~ x^n ~ e^x ~ - ~ n ~ I_{~ (n ~ - ~ 1) ~}(x) \\ \\
	\end{aligned}$ ~\\ 
	
% c)
	
	\item \hfill \break
	
	$\begin{aligned}[t]
	I_1(x) ~~ &= ~~ x ~ e^x ~ - ~ 1 ~ I_{~ (1 ~ - ~ 1) ~}(x) \\ \\
	&= ~~ x ~ e^x ~ - ~ I_{0}(x) \\ \\
	&= ~~ x ~ e^x ~ - ~ I_{0}(x) \\ \\
	&= ~~ x ~ e^x ~ - ~ \left( e^x ~ - ~ 1 \right) \\ \\
	&= ~~ x ~ e^x ~ - ~ e^x ~ + ~ 1 \\ \\
	&= ~~ \left( x ~ - 1 \right) ~ e^x ~ + ~ 1 \\ \\
	\end{aligned}$
	
	\newpage
	
	$\begin{aligned}[t]
	I_2(x) ~~ &= ~~ x^2 ~ e^x ~ - ~ 2 ~ I_{~ (2 ~ - ~ 1) ~}(x) \\ \\
	&= ~~ x^2 ~ e^x ~ - ~ 2 ~ I_{1}(x) \\ \\
	&= ~~ x^2 ~ e^x ~ - ~ 2 ~ \left( ~ \left( x ~ - 1 \right) ~ e^x ~ + ~ 1 ~ \right) \\ \\
	&= ~~ x^2 ~ e^x ~ - ~ 2 ~  \left( x ~ - 1 \right) ~ e^x ~ - ~ 2 \\ \\
	&= ~~ \left( ~ x^2 ~ - ~ 2 ~ \left( x ~ - 1 \right) ~ \right) ~ e^x ~ - ~ 2 \\ \\
	&= ~~ \left( ~ x^2 ~ - ~ 2 ~ x ~ + 2 ~ \right) ~ e^x ~ - ~ 2 \\ \\
	\end{aligned}$ \\
	
	~\\
	
	$\begin{aligned}[t]
	I_3(x) ~~ &= ~~ x^3 ~ e^x ~ - ~ 3 ~ I_{~ (3 ~ - ~ 1) ~}(x) \\ \\
	&= ~~ x^3 ~ e^x ~ - ~ 3 ~ I_{2}(x) \\ \\
	&= ~~ x^3 ~ e^x ~ - ~ 3 ~ \left( ~ \left( ~ x^2 ~ - ~ 2 ~ x ~ + 2 ~ \right) ~ e^x ~ - ~ 2 ~ \right) \\ \\
	&= ~~ x^3 ~ e^x ~ - ~ 3 ~ \left( ~ x^2 ~ - ~ 2 ~ x ~ + 2 ~ \right) ~ e^x ~ + ~ 6 \\ \\
	&= ~~ \left( ~ x^3 ~ - ~ 3 ~ \left( ~ x^2 ~ - ~ 2 ~ x ~ + 2 ~ \right) ~ \right) ~ e^x ~ + ~ 6 \\ \\
	&= ~~ \left( ~ x^3 ~ - ~ 3 ~ x^2 ~ + ~ 6 ~ x ~ - 6 ~ \right) ~ e^x ~ + ~ 6 \\ \\
	\end{aligned}$
	
	
	~\\
	
	\newpage
	
	\item Nach Aufgabe ~ c) ~ ist bereits ein Teilmuster: ~ $I_{n}(x) ~ = ~ ( ~ ... ~ ) ~ e^x ~ + ~ (-1)^{~ ...} ~ n!$ ~ zu vermuten. \\
	
\end{enumerate}

	
	~\\ Pochhammer-Symbol:
	
	~\\
	
	$\begin{aligned}[t]
	(a)_0 ~~ &:= ~~ 1 \\ \\
	(a)_n ~~ &:= ~~ a ~ (a ~ + ~ 1) ~ (a ~ + ~ 2) ~ ... ~ (a ~ + ~ n ~ - ~ 1) \\ \\ \\
	(a)_1 ~~ &~= ~~ a \\ \\
	(a)_2 ~~ &~= ~~ a ~ (a ~ + ~ 1) \\ \\
	(a)_3 ~~ &~= ~~ a ~ (a ~ + ~ 1) ~ (a ~ + ~ 2) \\ \\
	(a)_4 ~~ &~= ~~ a ~ (a ~ + ~ 1) ~ (a ~ + ~ 2) ~ (a ~ + ~ 3) \\ \\
	...
	\end{aligned}$

	~\\
	
	D.h. wir brauchen einen Index (z.B. von einem Summen-, Produkt- oder einem anderen -Zeichen mit Index). Stellt sich die Frage: Wie können wir ~ $I_n(x)$ ~ als Summe schreiben? Probieren wir die Beispiele ~ $I_0, ~ I_1, ~ I_2, ~ I_3$ ~ so zu schreiben, dass sich vielleicht ein Muster zeigt ...
	
	~\\
	
	\[ \left( ~ \sum_{k ~ = ~ 1}^{...} ~ ... ~ \right) ~ e^x ~ + ~ (-1)^{~ ...} ~ n! \]
	
	
	\newpage	

	$\begin{aligned}[t]
	I_0(x) ~~ &= ~~ e^x ~ - ~ 1 \\ \\
	&= ~~ \left( ~ 1 ~ x^0 ~ \right) ~ e^x ~ - ~ 1 \\ \\
	&= ~~ \left( ~ (-0)_0 ~ x^{~ 1 - 1} ~ \right) ~ e^x ~ + ~ \left( -1 \right)^{~ 0 + 1} ~ 0!
	\end{aligned}$
	
	~\\~\\
	
	$\begin{aligned}[t]
	I_1(x) ~~ &= ~~ \left( ~ x ~ - 1 ~ \right) ~ e^x ~ + ~ 1 \\ \\
	&= ~~ \left( ~ x^1 ~ - 1 ~ x^0 ~ \right) ~ e^x ~ + ~ \left( ~ 1 ~ \right) ~ 1 \\ \\
	&= ~~ \left( ~ (-1)_0 ~ x^{~ 2 - 1} ~ + ~ (-1)_1 ~ x^{~ 1 - 1} ~ \right) ~ e^x ~ + ~ \left( -1 \right)^{~ 1 + 1} ~ 1!
	\end{aligned}$
	
	~\\~\\
	
	$\begin{aligned}
	I_2(x) ~~ &= ~~ \left( ~ x^2 ~ - ~ 2 ~ x ~ + 2 ~ \right) ~ e^x ~ - ~ 2 \\ \\
	&= ~~ \left( ~ x^2 ~ - ~ 2 ~ x^1 ~ + 2 ~ x^0 ~ \right) ~ e^x ~ + ~ \left( ~ -1 ~ \right) ~ 2 \\ \\
	&= ~~ \left( ~ (-2)_0 ~ x^{~ 3 - 1} ~ + ~ (-2)_1 ~ x^{~ 2 - 1} ~ + ~ (-2)_2 ~ x^{~ 1 - 1} ~ \right) ~ e^x ~ + ~ \left( -1 \right)^{~ 2 + 1} ~ 2!
	\end{aligned}$
	
	~\\~\\
	
	$\begin{aligned}[t]
	I_3(x) ~~ &= ~~ \left( ~ x^3 ~ - ~ 3 ~ x^2 ~ + ~ 6 ~ x ~ - 6 ~ \right) ~ e^x ~ + ~ \left( ~ 1 ~ \right) ~ 6 \\ \\
	&= ~~ \left( ~ x^3 ~ - ~ 3 ~ x^2 ~ + ~ 6 ~ x^1 ~ - 6 ~ x^0 ~ \right) ~ e^x ~ + ~ \left( ~ 1 ~ \right) ~ 6 \\ \\
	&= ~~ \left( ~ (-3)_0 ~ x^{~ 4 - 1} ~ + ~ (-3)_1 ~ x^{~ 3 - 1} ~ + ~ (-3)_2 ~ x^{~ 2 - 1} ~ + ~ (-3)_3 ~ x^{~ 1 - 1} ~ \right) \\ \\
	& ~~~~ \cdot ~ e^x ~ + ~ \left( -1 \right)^{~ 3 + 1} ~ 3!
	\end{aligned}$
	
	~\\~\\
	
	$\begin{aligned}[t]
	I_4(x) ~~ &= ~~ x^4 ~ e^{x} ~ - ~ 4 ~ I_{~ 4 - 1 ~}(x) \\ \\
	&= ~~ x^4 ~ e^{x} ~ - ~ 4 ~ I_{3}(x) \\ \\
	&= ~~ x^4 ~ e^{x} ~ - ~ 4 ~ \left( ~ x^3 ~ - ~ 3 ~ x^2 ~ + ~ 6 ~ x ~ - 6 ~ \right) \\ \\
	&= ~~ x^4 ~ e^{x} ~ - ~ 4 ~ \left( ~ \left( ~ x^3 ~ - ~ 3 ~ x^2 ~ + ~ 6 ~ x ~ - 6 ~ \right) ~ e^x ~ - ~ 6 ~ \right) \\ \\
	&= ~~ x^4 ~ e^{x} ~ - ~ 4 ~ \left( ~ x^3 ~ - ~ 3 ~ x^2 ~ + ~ 6 ~ x ~ - 6 ~ \right) ~ e^x ~ + ~ 24 \\ \\
	&= ~~ \left( ~ x^4 ~ - ~ 4 ~ \left( ~ x^3 ~ - ~ 3 ~ x^2 ~ + ~ 6 ~ x ~ - 6 ~ \right) ~ \right) ~ e^x ~ + ~ 24 \\ \\
	&= ~~ \left( ~ x^4 ~ - ~ 4 ~ x^3 ~- ~ 12 ~ x^2 ~ - ~ 24 ~ x ~ + 24 ~ \right) ~ e^x ~ + ~ 24 \\ \\
	&= ~~ \left( ~ x^4 ~ - ~ 4 ~ x^3 ~- ~ 12 ~ x^2 ~ - ~ 24 ~ x^1 ~ + 24 ~ x^0 ~ \right) ~ e^x ~ + ~ 24
	\end{aligned}$
	
	~\\
	
	\newpage
	
	Behauptung: \\
	
	\[ \forall n ~ \in ~ \mathbb{N}: ~~ I_{~ (n ~ - ~ 1) ~} (x) ~~ = ~~ e^x ~ \sum_{k ~ = ~ 1}^{n} ~ \left( ~ -n ~ + ~ 1 ~ \right)_{(n ~ - ~ k)} ~ x^{~ k ~ - ~ 1} ~ + ~ (-1)^n ~ (n ~ - ~ 1)! \]
	
	~\\
	~\\
	
	Beweis durch vollständige Induktion.
	
	~\\
	
	\begin{tabularx}{\textwidth}{rX}
		
		Hinweis:  & Ich werde den Beweis ohne Voraussetzungen führen. Denn, wer ihn oder die Eigenschaften des \textit{Pochhammer-Symbols} kennt, wird vielleicht \textit{tricksen}. \\
		
	\end{tabularx}
	
	~\\
	~\\
	
	I.A.:
	
	~\\
	
$\begin{aligned}[t]
	n ~ = ~ 1 ~ : \qquad &\textcolor{white}{=} ~~ I_{~ (1 ~ - ~ 1) ~} (x) \\ \\
	&= ~~ I_0(x) \\ \\
	&= ~~ e^x ~ - ~ 1 \\ \\
	&\oset[1.5ex]{?}{=} ~~ e^x ~ \sum_{k ~ = ~ 1}^{1} ~ \left( ~ -1 ~ + ~ 1 ~ \right)_{(1 ~ - ~ k)} ~ x^{~ k ~ - ~ 1} ~ + ~ (-1)^1 ~ (1 ~ - ~ 1)! \\ \\
	&= ~~ e^x ~ \left( ~ 0 ~ \right)_{(0)} ~ x^{~ 0} ~ - ~ 0! \\ \\
	&= ~~ e^x ~ - ~ 1 \\ \\
	&\square
\end{aligned}$

	\newpage
	
$\begin{aligned}[t]
n ~ = ~ 2 ~ : \qquad &\textcolor{white}{=} ~~ I_{~ (2 ~ - ~ 1) ~} (x) \\ \\
&= ~~ I_1(x) \\ \\
&= ~~ \left( ~ - 1 ~ + ~ x ~ \right) ~ e^x ~ + ~ 1 \\ \\
&\oset[1.5ex]{?}{=} ~~ e^x ~ \sum_{k ~ = ~ 1}^{2} ~ \left( ~ -2 ~ + ~ 1 ~ \right)_{(2 ~ - ~ k)} ~ x^{~ k ~ - ~ 1} ~ + ~ (-1)^2 ~ (2 ~ - ~ 1)! \\ \\
&= ~~ e^x ~ \sum_{k ~ = ~ 1}^{2} ~ \left( ~ -1 ~ \right)_{(2 ~ - ~ k)} ~ x^{~ k ~ - ~ 1} ~ + ~ 1! \\ \\
&= ~~ e^x ~ \left( ~ \left( ~ -1 ~ \right)_{(2 ~ - ~ 1)} ~ x^{~ 1 ~ - ~ 1} ~ + ~ \left( ~ -1 ~ \right)_{(2 ~ - ~ 2)} ~ x^{~ 2 ~ - ~ 1} ~ \right) ~ + ~ 1  \\ \\
&= ~~ e^x ~ \left( ~ \left( ~ -1 ~ \right)_{(1)} ~ x^{~ 0} ~ + ~ \left( ~ -1 ~ \right)_{(0)} ~ x^{~ 1} ~ \right) ~ + ~ 1  \\ \\
&= ~~ e^x ~ \left( ~ -1 ~ + ~ x ~ \right) ~ + ~ 1  \\ \\
&\square
\end{aligned}$

~\\
~\\

I.H.:

\[ \exists n ~ \in ~ \mathbb{N}: ~~ I_{~ (n ~ - ~ 1) ~} (x) ~~ = ~~ e^x ~ \sum_{k ~ = ~ 1}^{n} ~ \left( ~ -n ~ + ~ 1 ~ \right)_{(n ~ - ~ k)} ~ x^{~ k ~ - ~ 1} ~ + ~ (-1)^n ~ (n ~ - ~ 1)! \]
		
	\newpage
	
	I.S.: \setcounter{tc}{0}
	
\begin{flalign*}
%
% \im & & <left_expression> &= <right_expression>
%
& & I_{~ ( ~ (n ~ + ~ 1) ~ - ~ 1 ~ ) ~} (x) ~~ &= ~~ e^x ~ \sum_{k ~ = ~ 1}^{n ~ + ~ 1} ~ \left( ~ -(n ~ + ~ 1) ~ + ~ 1 ~ \right)_{( ~ (n ~ + ~ 1) ~ - ~ k ~ )} ~ x^{~ k ~ - ~ 1} \\ \\
& & & ~~~ ~ + ~ (-1)^{~ (n ~ + ~ 1)} ~ \left( ~ (n ~ + ~ 1) ~ - ~ 1 ~ \right)! \\ \\ \\
%
\im \quad & & I_n (x) ~~ &= ~~ e^x ~ \sum_{k ~ = ~ 1}^{n ~ + ~ 1} ~ \left( ~ -n ~ \right)_{( ~ n ~ + ~ 1 ~ - ~ k ~ )} ~ x^{~ k ~ - ~ 1} \\ \\
& & & ~~~ ~ + ~ (-1)^{~ (n ~ + ~ 1)} ~ n! \\ \\ \\
%
\im \quad & & x^n ~ e^x ~ - ~ n ~ I_{~ (n ~ - ~ 1) ~} (x) ~~ &= ~~ e^x ~ \sum_{k ~ = ~ 1}^{n ~ + ~ 1} ~ \left( ~ -n ~ \right)_{( ~ n ~ + ~ 1 ~ - ~ k ~ )} ~ x^{~ k ~ - ~ 1} \\ \\
& & & ~~~ ~ + ~ (-1)^{~ (n ~ + ~ 1)} ~ n! \\ \\ \\
%
\im \quad & & 0 ~~ &= ~~ e^x ~ \sum_{k ~ = ~ 1}^{n ~ + ~ 1} ~ \left( ~ -n ~ \right)_{( ~ n ~ + ~ 1 ~ - ~ k ~ )} ~ x^{~ k ~ - ~ 1} \\ \\
& & & ~~~ ~ - ~ x^n ~ e^x ~ + ~ n ~ I_{~ (n ~ - ~ 1) ~} (x) \\ \\
& & & ~~~ ~ + ~ (-1)^{~ (n ~ + ~ 1)} ~ n! \\ \\ \\
%
\end{flalign*}


\begin{flalign*}
%
% \im & & <left_expression> &= <right_expression>
%
\im \quad & & 0 ~~ &= ~~ e^x ~ \left( ~ (~ -(n ~ + ~ 1) ~)_{(~ n ~ + ~ 1 ~ -(n ~ + ~ 1) ~)} ~ + ~ \sum_{k ~ = ~ 1}^{n} ~ \left( ~ -n ~ \right)_{( ~ n ~ + ~ 1 ~ - ~ k ~ )} ~ x^{~ k ~ - ~ 1} ~ \right) \\ \\
& & & ~~~ ~ - ~ x^n ~ e^x ~ + ~ n ~ I_{~ (n ~ - ~ 1) ~} (x) \\ \\
& & & ~~~ ~ + ~ (-1)^{~ (n ~ + ~ 1)} ~ n! \\ \\ \\
%
\im \quad & & 0 ~~ &= ~~ e^x ~ \left( ~ x^n ~ + ~ \sum_{k ~ = ~ 1}^{n} ~ \left( ~ -n ~ \right)_{( ~ n ~ + ~ 1 ~ - ~ k ~ )} ~ x^{~ k ~ - ~ 1} ~ \right) \\ \\
& & & ~~~ ~ - ~ x^n ~ e^x ~ + ~ n ~ I_{~ (n ~ - ~ 1) ~} (x) \\ \\
& & & ~~~ ~ + ~ (-1)^{~ (n ~ + ~ 1)} ~ n! \\ \\ \\
%
\im \quad & & 0 ~~ &= ~~ e^x ~ x^n ~ + ~ e^x ~ \sum_{k ~ = ~ 1}^{n} ~ \left( ~ -n ~ \right)_{( ~ n ~ + ~ 1 ~ - ~ k ~ )} ~ x^{~ k ~ - ~ 1} \\ \\
& & & ~~~ ~ - ~ x^n ~ e^x ~ + ~ n ~ I_{~ (n ~ - ~ 1) ~} (x) \\ \\
& & & ~~~ ~ + ~ (-1)^{~ (n ~ + ~ 1)} ~ n! \\ \\ \\
%
\im \quad & & 0 ~~ &= ~~ e^x ~ \sum_{k ~ = ~ 1}^{n} ~ \left( ~ -n ~ \right)_{( ~ n ~ + ~ 1 ~ - ~ k ~ )} ~ x^{~ k ~ - ~ 1} \\ \\
& & & ~~~ ~ + ~ n ~ I_{~ (n ~ - ~ 1) ~} (x) \\ \\
& & & ~~~ ~ + ~ (-1)^{~ (n ~ + ~ 1)} ~ n!
%
\end{flalign*}


\begin{flalign*}
%
% \im & & <left_expression> &= <right_expression>
%
\im \quad & & 0 ~~ &= ~~ e^x ~ \sum_{k ~ = ~ 1}^{n} ~ \left( ~ -n ~ \right)_{( ~ n ~ + ~ 1 ~ - ~ k ~ )} ~ x^{~ k ~ - ~ 1} \\ \\
& & & ~~~ ~ + ~ n ~ \left( ~ e^x ~ \sum_{k ~ = ~ 1}^{n} ~ \left( ~ -n ~ + ~ 1 ~ \right)_{(n ~ - ~ k)} ~ x^{~ k ~ - ~ 1} ~ + ~ (-1)^n ~ (n ~ - ~ 1)! ~ \right) \\ \\
& & & ~~~ ~ + ~ (-1)^{~ (n ~ + ~ 1)} ~ n! \\ \\ \\
%
\im \quad & & 0 ~~ &= ~~ e^x ~ \sum_{k ~ = ~ 1}^{n} ~ \left( ~ -n ~ \right)_{( ~ n ~ + ~ 1 ~ - ~ k ~ )} ~ x^{~ k ~ - ~ 1} \\ \\
& & & ~~~ ~ + ~ e^x ~ \sum_{k ~ = ~ 1}^{n} ~ n ~ \left( ~ -n ~ + ~ 1 ~ \right)_{(n ~ - ~ k)} ~ x^{~ k ~ - ~ 1} ~ + ~ (-1)^n ~ (n ~ - ~ 1)! ~ n \\ \\
& & & ~~~ ~ + ~ (-1)^{~ (n ~ + ~ 1)} ~ n! \\ \\ \\
%
\im \quad & & 0 ~~ &= ~~ e^x ~ \sum_{k ~ = ~ 1}^{n} ~ \left( ~ -n ~ \right)_{( ~ n ~ + ~ 1 ~ - ~ k ~ )} ~ x^{~ k ~ - ~ 1} \\ \\
& & & ~~~ ~ + ~ e^x ~ \sum_{k ~ = ~ 1}^{n} ~ n ~ \left( ~ -n ~ + ~ 1 ~ \right)_{(n ~ - ~ k)} ~ x^{~ k ~ - ~ 1} ~ + ~ (-1)^n ~ n! \\ \\
& & & ~~~ ~ + ~ (-1)^{~ (n ~ + ~ 1)} ~ n! \\ \\ \\
%
\im \quad & & 0 ~~ &= ~~ e^x ~ \sum_{k ~ = ~ 1}^{n} ~ \left( ~ -n ~ \right)_{( ~ n ~ + ~ 1 ~ - ~ k ~ )} ~ x^{~ k ~ - ~ 1} \\ \\
& & & ~~~ ~ + ~ e^x ~ \sum_{k ~ = ~ 1}^{n} ~ n ~ \left( ~ -n ~ + ~ 1 ~ \right)_{(n ~ - ~ k)} ~ x^{~ k ~ - ~ 1} ~ + ~ (-1) ~ (-1)^{~ (n ~ - ~ 1)} ~ n! \\ \\
& & & ~~~ ~ + ~ (-1)^{~ (n ~ + ~ 1)} ~ n!
%
\end{flalign*}


\begin{flalign*}
%
% \im & & <left_expression> &= <right_expression>
%
\im \quad & & 0 ~~ &= ~~ e^x ~ \sum_{k ~ = ~ 1}^{n} ~ \left( ~ -n ~ \right)_{( ~ n ~ + ~ 1 ~ - ~ k ~ )} ~ x^{~ k ~ - ~ 1} ~ + ~ e^x ~ \sum_{k ~ = ~ 1}^{n} ~ n ~ \left( ~ -n ~ + ~ 1 ~ \right)_{(n ~ - ~ k)} ~ x^{~ k ~ - ~ 1} \\ \\
%
\im \quad & & 0 ~~ &= ~~ e^x ~ \left( ~ \sum_{k ~ = ~ 1}^{n} ~ \left( ~ -n ~ \right)_{( ~ n ~ + ~ 1 ~ - ~ k ~ )} ~ x^{~ k ~ - ~ 1} ~ + ~ \sum_{k ~ = ~ 1}^{n} ~ n ~ \left( ~ -n ~ + ~ 1 ~ \right)_{(n ~ - ~ k)} ~ x^{~ k ~ - ~ 1} ~ \right) \\ \\
%
\im \quad & & 0 ~~ &= ~~ e^x ~ \left( ~ \sum_{k ~ = ~ 1}^{n} ~ \left( ~ -n ~ \right)_{( ~ n ~ + ~ 1 ~ - ~ k ~ )} ~ x^{~ k ~ - ~ 1} ~ - ~ \sum_{k ~ = ~ 1}^{n} ~ \underbrace{(-n) ~ \left( ~ -n ~ + ~ 1 ~ \right)_{(n ~ - ~ k)}}_{~ ? ~} ~ x^{~ k ~ - ~ 1} ~ \right) \\ \\
%
\im \quad & & 0 ~~ &= ~~ e^x ~ \left( ~ \sum_{k ~ = ~ 1}^{n} ~ \left( ~ -n ~ \right)_{( ~ n ~ + ~ 1 ~ - ~ k ~ )} ~ x^{~ k ~ - ~ 1} ~ - ~ \sum_{k ~ = ~ 1}^{n} ~ \left( ~ -n ~ \right)_{(n ~ + ~ 1 ~ - ~ k)} ~ x^{~ k ~ - ~ 1} ~ \right) \\ \\
%
\im \quad & & 0 ~~ &= ~~ 0 \\ \\
%
%\im \quad & & 0 ~~ &= ~~ e^x ~ \left( ~ \sum_{k ~ = ~ 1}^{n} ~ \left( ~ -n ~ \right)_{( ~ n ~ + ~ 1 ~ - ~ k ~ )} ~ x^{~ k ~ - ~ 1} ~ + ~ n ~ \left( ~ -n ~ + ~ 1 ~ \right)_{(n ~ - ~ k)} ~ x^{~ k ~ - ~ 1} ~ \right) \\ \\
%
%\im \quad & & 0 ~~ &= ~~ e^x ~ \left( ~ \sum_{k ~ = ~ 1}^{n} ~ \left( ~ -n ~ \right)_{( ~ n ~ + ~ 1 ~ - ~ k ~ )} ~ x^{~ k ~ - ~ 1} ~ + ~ n ~ \left( ~ -n ~ + ~ 1 ~ \right)_{(n ~ - ~ k)} ~ x^{~ k ~ - ~ 1} ~ \right) \\ \\
%
%\im \quad & & 0 ~~ &= ~~ e^x ~ \left( ~ \sum_{k ~ = ~ 1}^{n} ~ \underbrace{\left( ~ \left( ~ -n ~ \right)_{( ~ n ~ + ~ 1 ~ - ~ k ~ )} ~ + ~ n ~ \left( ~ -n ~ + ~ 1 ~ \right)_{(n ~ - ~ k)} ~ \right)}_{ \uset[1.5ex]{?}{=} ~ 0} ~ x^{~ k ~ - ~ 1} ~ \right) \\ \\
%%
\end{flalign*}

~\\

Bemerkung: \setcounter{tc}{0}

\begin{align*}
	&~ \qquad (-n) ~ \left( ~ -n ~ + ~ 1 ~ \right)_{(n ~ - ~ k)} ~~ = ~~ \left( ~ -n ~ \right)_{( ~ n ~ + ~ 1 ~ - ~ k ~ )} \\ \\
	&\eqv \qquad -n ~ \prod_{i ~ = ~ 0}^{n ~ - ~ k ~ - 1} ~ -n ~ + ~ 1 ~ + ~ i ~~ = ~~ \prod_{i ~ = ~ 0}^{n ~ - ~ k ~ - ~ 1 ~ + ~ 1 ~ = ~ n ~ - ~ k} ~ -n ~ + ~ i
\end{align*}

~\\

$\blacksquare$

~\\
~\\

Übrigens: Das Pochhammersymbol wird 0, sobald der Index vom Argument abhängt, diesen um mindestens 1 übersteigt und die Vorzeichen unterschiedlich sind, ist eine 0 im Produkt. Wenn es das noch nicht gibt, könnte man dazu Pochhammer-Null sagen.

	
	
	\newpage
	
	~\\
	~\\
	
	
	\newpage
	
	%Nach unermüdlichen Lösungsangriffen:
	
	Notiz/Gedanken: Hier zeigt sich wieder einmal die Gefahr der Pünktchen- und Kurzschreibweisen. Vielleicht vom Aufgabensteller gut gemeint, oder frech (?) ist die Beschreibung des Pochhammer-Symbols.
	Zunächst denkt man nicht weiter darüber nach, als sich in etwa Vorzustellen, wie die endlichen Produkte aussehen. Am Ende der Induktion bemerkt man, dass sich der Beweis nicht sauber lösen lässt, wenn man sich nicht eine saubere Definition des Pochhammer-Symbols überlegt und zeigt, dass die Differenz 0 ergibt. Man hat gar nichts von der Kurzschreibweise, da man im letzten Schritt genauso viel Zeit verbraucht darüber nachzudenken, wie wenn man gleich ein Produktzeichen eingeführt hätte! Im allgemeinen lohnen sich natürlich Kurzschreibweisen, nur hier in der Aufgabe eben nicht. Was ja auch Teil der Aufgabe gewesen sein kann.



%	&= ~~ \left( ~ x ~ \left( ~ x^3 ~ - ~ 4 ~ x^2 ~- ~ 12 ~ x ~ - ~ 24 ~ \right) ~ + 24 ~ \right) ~ ... \\ \\
%	&= ~~ \left( ~ x ~ \left( ~ x ~ \left( x^2 ~ - ~ 4 ~ x ~- ~ 12 ~ \right) ~ - ~ 24 ~ \right) ~ + 24 ~ \right) ~ ... \\ \\
%	&= ~~ \left( ~ x ~ \left( ~ x ~ \left( ~ x ~ \left( x ~ - ~ 4 ~ \right) ~ - ~ 12 ~ \right) ~ - ~ 24 ~ \right) ~ + 24 ~ \right) ~ ... \\ \\

% % % % % % % % % % % % % % % % % % % % % % % % % % % % % % % % % % % % % % % % % % %


\newpage


% 4

\section{~}

~\\

\begin{enumerate}[leftmargin=*, labelsep=2em, itemsep=3em, label=\alph*)]


% a)

	\item $\begin{aligned}[t]
	\int_{x_0}^{x_1} ~ dx ~ f(x) ~~ &= ~~ \int_{x_0}^{x_1} ~ dx ~ \frac{1}{y(x)} ~ \dydx \\ \\
	&= ~~ {\begin{cases}
		~ u ~ := ~ y(x) ~
	\end{cases}} : \qquad \int_{x_0}^{x_1} ~ dx ~ \frac{1}{u} ~ \dudx \\ \\
	&= ~~ {\begin{cases}
		~ du ~ = ~  dx ~ \dudx ~
	\end{cases}} : \qquad \int_{x_0}^{x_1} ~ du ~ \frac{1}{u} \\ \\
	&= ~~ {\big[ ln ~ \left| ~ u ~ \right| \big]}_{y(x_0)}^{y(x_1)} \\ \\
	&= ~~ ln ~ \left| ~ y(x_1) ~ \right| ~ - ~ ln ~ \left| ~ y(x_0) ~ \right| \\ \\
	&= ~~ ln ~ \frac{ \left| ~ y(x_1) ~ \right| }{ \left| ~ y(x_0) ~ \right| } \qquad, ~~ \text{da ~} y(x_0) ~ \neq ~ 0 \\ \\
	&= ~~ ln ~ \left| ~ \frac{ y(x_1) }{ y(x_0) } ~ \right| \\ \\
	&= ~~ \begin{cases}
		~ sgn ~~ y(x_0) ~ = ~ sgn ~~ y(x_1) ~
	\end{cases} \Rightarrow \qquad ln ~ \frac{ y(x_1) }{ y(x_0) }  \\ \\
	\end{aligned}$
	
	
% b)
	
	\item 
	
	\setcounter{tc}{0}
	
	$\begin{aligned}[t]
	&\textcolor{white}{\Leftrightarrow} ~~ \int_{x_0}^{x_1} ~ f(x) ~~ = ~~ ln ~ \frac{~ y \left( x_1 \right) ~}{~ y \left( x_0 \right) ~} \\ \\
	&\im ~~ {F\left(x\right)|}_{x_0}^{x_1} ~~ = ~~ ln ~ \frac{~ y \left( x_1 \right) ~}{~ y \left( x_0 \right) ~} \\ \\
	&\im ~~ F(x_1) ~ - ~ F(x_0) ~~ = ~~ ln ~ \frac{~ y \left( x_1 \right) ~}{~ y \left( x_0 \right) ~} \\ \\
	&\im ~~ e^{~ F(x_1) ~ - ~ F(x_0)} ~~ = ~~ e^{ ~ ln ~ \frac{~ y \left( x_1 \right) ~}{~ y \left( x_0 \right) ~} } \\ \\
	&\im ~~ e^{~ F(x_1) ~ - ~ F(x_0)} ~~ = ~~ \frac{~ y \left( x_1 \right) ~}{~ y \left( x_0 \right) ~} \\ \\
	&\im ~~ y \left( x_0 \right) ~ e^{~ F(x_1) ~ - ~ F(x_0)} ~~ = ~~ y \left( x_1 \right) \\ \\
	\end{aligned}$
	
	~\\
	
	\setcounter{tc}{0}
	
	\begin{longtable}[l]{l@{\hspace{3em}}l}
		
		\itc, \itc & Definition: bestimmtes Integral. \\ \\
		\itc, \itc & Anwendung: Exponentialfunktion. \\ \\
		\itc & $y \left( x_0 \right)$ ~ nach links. \\ \\
		\itc & ...?
		
	\end{longtable}
	
	
	\newpage
	

% c)

	\item
	
	$y(x) ~ = ~ 0 \quad : ~~ \ddy ~ 0 ~ = ~ \lambda ~ x^{\alpha} ~ 0 ~~ \Rightarrow ~~ 0 ~ = ~ 0$
	
	~\\
	
	\setcounter{tc}{0}
	
	$\begin{aligned}[t]
		&\textcolor{white}{\Leftrightarrow} ~~ \dydx ~~ = ~~ \lambda ~ x^{\alpha} ~ y(x) \\ \\
		&\im ~~ {\begin{cases}
			~ y(x) ~ \neq ~ 0 ~
		\end{cases}} : \qquad \dydx ~ \frac{1}{y(x)} ~~ = ~~ \lambda ~ x^{\alpha} \\ \\
		&\im ~~ \idx ~ \dydx ~ \frac{1}{y(x)} ~~ = ~~ \lambda ~ \idx ~ x^{\alpha} \\ \\
		&\im ~~ {\begin{cases}
			~ u ~ := ~ y(x) ~
			\end{cases}} : \qquad \idx ~ \dudx ~ \frac{1}{u} ~~ = ~~ \lambda ~ \idx ~ x^{\alpha} \\ \\
		&\im ~~ {\begin{cases}
			~ du ~ = ~ dx ~ \dudx ~
			\end{cases}} : \qquad \idu ~ \frac{1}{u} ~~ = ~~ \lambda ~ \idx ~ x^{\alpha} \\ \\
		&\im ~~ ln ~ \left| ~ u ~ \right| ~ + ~ C_1 ~~ = ~~ \lambda ~ \idx ~ x^{\alpha} \qquad, ~~ C_1 ~ \in ~ \mathbb{K} \\ \\
		&\im ~~ ln ~ \left| ~ y(x) ~ \right| ~ + ~ C_1 ~~ = ~~ \lambda ~ \idx ~ x^{\alpha} \\ \\
		&\im ~~ ln ~ \left| ~ y(x) ~ \right| ~ + ~ C_1 ~~ = ~~ \lambda ~ \left( \frac{x^{\alpha ~ + ~ 1}}{\alpha ~ + ~ 1} ~ + ~ C_2 \right) \qquad, ~~ C_2 ~ \in ~ \mathbb{K} ~~, ~~ \alpha ~ \neq ~ -1 \\ \\
		&\im ~~ ln ~ \left| ~ y(x) ~ \right| ~~ = ~~ \lambda ~ \frac{x^{\alpha ~ + ~ 1}}{\alpha ~ + ~ 1} ~ + ~ \lambda ~ C_2 ~ - ~ C_1 \\ \\
		&\im ~~ {\begin{cases}
			~ C_{(1)}(\lambda) ~ := ~ \lambda ~ C_2 ~ - ~ C_1 ~~ ~ \in ~ \mathbb{K} ~
		\end{cases}} : \qquad ln ~ \left| ~ y(x) ~ \right| ~~ = ~~ \lambda ~ \frac{x^{\alpha ~ + ~ 1}}{\alpha ~ + ~ 1} ~ + ~ C_{(1)}(\lambda) \\ \\
		&\im ~~ e^{~ ln ~ \left| ~ y(x) ~ \right|} ~~ = ~~ e^{\lambda ~ \frac{x^{\alpha ~ + ~ 1}}{\alpha ~ + ~ 1} ~ + ~ C_{(1)}(\lambda) } \\ \\
		&\im ~~ \left| ~ y(x) ~ \right| ~~ = ~~ e^{~ \lambda ~ \frac{x^{\alpha ~ + ~ 1}}{\alpha ~ + ~ 1} ~ + ~ C_{(1)}(\lambda) }
	\end{aligned}$
	
	
	\newpage
	
	
	$\begin{aligned}
		&\im ~~ y(x) ~~ = ~~ \pm ~ e^{~ \lambda ~ \frac{x^{\alpha ~ + ~ 1}}{\alpha ~ + ~ 1} ~ + ~ C_{(1)}(\lambda) } \\ \\
		&\im ~~ y(x) ~~ = ~~ \pm ~ e^{~ \lambda ~ \frac{x^{\alpha ~ + ~ 1}}{\alpha ~ + ~ 1}} ~ e^{C_{(1)}(\lambda) } \\ \\
		&\im ~~ {\begin{cases}
			~ C_{(2)}(\lambda) ~ := ~ e^{ C_{(1)}(\lambda) } ~~ ~ \in ~ \mathbb{K} ~
			\end{cases}} : \qquad y(x) ~~ = ~~ \pm ~ e^{~ \lambda ~ \frac{x^{\alpha ~ + ~ 1}}{\alpha ~ + ~ 1}} ~ C_{(2)}(\lambda) \\ \\
		&\im ~~ y(x) ~~ = ~~ \pm ~ C_{(2)}(\lambda) ~ e^{~ \lambda ~ \frac{x^{\alpha ~ + ~ 1}}{\alpha ~ + ~ 1} } \\ \\ \\ % \\
		&\im ~~ {\begin{cases}
			~ \alpha ~ = ~ -1 ~
		\end{cases}} : \qquad ln ~ \left| ~ y(x) ~ \right| ~ + ~ C_1 ~~ = ~~ \lambda ~ \idx ~ \frac{1}{x} \\ \\
		&\im ~~ {\begin{cases}
		~ x ~ > ~ 0 ~
		\end{cases}} : \qquad ln ~ \left| ~ y(x) ~ \right| ~ + ~ C_1 ~~ = ~~ \lambda ~ \left( ~ ln ~ x ~ + ~ C_3 ~ \right) \qquad, ~~ C_3 ~ \in ~ \mathbb{K} \\ \\
		&\im ~~ ln ~ \left| ~ y(x) ~ \right| ~ + ~ C_1 ~~ = ~~ \lambda ~  ~ ln ~ x ~ + ~ \lambda ~ C_3 \\ \\
		&\im ~~ ln ~ \left| ~ y(x) ~ \right| ~~ = ~~ \lambda ~  ~ ln ~ x ~ + ~ \lambda ~ C_3 ~ - ~ C_1 \\ \\
		&\im ~~ {\begin{cases}
			~ C_{(3)}(\lambda) ~ := ~ \lambda ~ C_3 ~ - ~ C_1 ~~ ~ \in ~ \mathbb{K} ~
		\end{cases}} : \qquad ln ~ \left| ~ y(x) ~ \right| ~~ = ~~ \lambda ~  ~ ln ~ x ~ + ~ C_{(3)}(\lambda) \\ \\
		&\im ~~ e^{~ ln ~ \left| ~ y(x) ~ \right|} ~~ = ~~ e^{~ \lambda ~  ~ ln ~ x ~ + ~ C_{(3)}(\lambda)} \\ \\
		&\im ~~ \left| ~ y(x) ~ \right| ~~ = ~~ e^{~ \lambda ~  ~ ln ~ x ~ + ~ C_{(3)}(\lambda)} \\ \\
		&\im ~~ y(x) ~~ = ~~ \pm ~ e^{~ \lambda ~  ~ ln ~ x ~ + ~ C_{(3)}(\lambda)} \\ \\
		&\im ~~ y(x) ~~ = ~~ \pm ~ e^{~ \lambda ~  ~ ln ~ x } ~ e^{~ C_{(3)}(\lambda) } \\ \\
		&\im ~~ y(x) ~~ = ~~ \pm ~ x^{~ \lambda } ~ e^{~ C_{(3)}(\lambda) } \\ \\
		&\im ~~ {\begin{cases}
			~ C_{(4)}(\lambda) ~ := ~ e^{ ~ C_{(3)}(\lambda) } ~~ ~ \in ~ \mathbb{K} ~
		\end{cases}} : \qquad y(x) ~~ = ~~ \pm ~ x^{~ \lambda } ~ C_{(4)}(\lambda) \\ \\
	\end{aligned}$
	
	\newpage
	
	$\begin{aligned}
		&\im ~~ y(x) ~~ = ~~ \pm ~ C_{(4)}(\lambda) ~ x^{~ \lambda }
	\end{aligned}$
	
	~\\	
	~\\
	
	$\begin{aligned}
		&\Rightarrow \qquad \dydx ~ = ~ \lambda ~ x^{\alpha} ~ y(x) \\ \\
		&\Rightarrow ~~ {\begin{cases}
			~ y \left( x \right) ~ \neq ~ 0 ~ : & y \left( x \right) ~ = {\begin{cases}
				~ \alpha ~ \neq ~ 0 ~ : & \pm ~ C ~ e^{~ \lambda ~ \frac{x^{\alpha ~ + ~ 1}}{\alpha ~ + ~ 1} } \\ \\
				~ \alpha ~ = ~ 0 ~ : & \pm ~ C ~ x^{~ \lambda }
				\end{cases}} \\ \\
			~ y \left( x \right) ~ = ~ 0
		\end{cases}} \qquad, ~~ C ~ \in ~ \mathbb{K}
	\end{aligned}$
	
	~\\
	~\\
	
	Hinweis: Konstanten in Zukunft ohne Abhängigkeiten schreiben (unsauber, schneller, eigtl. will man ja auch nicht mehr, als auszudrücken, dass da eine Körperkonstante gewählt werden darf).
	TODO: Gilt die Rückrichtung?
	
	(TODO: Proben)
	
	
	~\\
	~\\
	
	
\setcounter{tc}{0}

\begin{longtable}{l@{\hspace{3em}}l}
	
	\itc & Trennung der Variablen ~ $y(x)$ ~ und ~ $x$. \\ \\
	\itc & ...
	
\end{longtable}

~\\	
~\\ Ich lasse hier ~ $\dydx$ ~ so stehen und schreibe nicht ~ $\dydx ~ = ~ \ddx ~ y(x)$ ~ aus. \\ Diese Kurzschreibweise lohnt sich. Gibt es Fälle, wo es nicht so ist?

~\\	Um die Merkregel der Integral-Substitution gleich sehen zu können, ist es besser, beim Trennen der Symbole ~ $y$ ~ und ~ $x$ ~ den Ausdruck ~ $\dydx$ ~ neben ~ $dx$ ~ des Integrals für die später folgende Integration zu schreiben. Daher: Variable, hier ~ $\frac{1}{y(x)}$ ~ gleich von rechts dran multiplizieren. Stimmt das immer?

~\\
	
	
	
	
	\newpage
	
	
	
	
	% d)
	
	\item

	$y(x) ~ = ~ 0 \quad : ~~ \ddy ~ 0 ~ = ~ exp(\alpha ~ x) ~ 0 ~~ \Rightarrow ~~ 0 ~ = ~ 0$
	
	~\\

	\setcounter{tc}{0}

	$\begin{aligned}[t]
		&\textcolor{white}{\Leftrightarrow} ~~ \dydx ~~ = ~~ exp(\alpha ~ x) ~ y(x) \\ \\
		&\im ~~ {\begin{cases}
			~ y(x) ~ \neq ~ 0 ~
			\end{cases}} : \qquad ~ \dydx ~ \frac{1}{y(x)} ~~ = ~~ e^{~ \alpha ~ x} \\ \\
		&\im ~~ \idx ~ \dydx ~ \frac{1}{y(x)} ~~ = ~~ \idx ~ e^{~ \alpha ~ x} \\ \\
		&\im ~~ {\begin{cases}
			~ u ~ := ~ y(x) ~
			\end{cases}} : \qquad \idx ~ \dudx ~ \frac{1}{u} ~~ = ~~ \idx ~ e^{~ \alpha ~ x} \\ \\
		&\im ~~ {\begin{cases}
			~ du ~ = ~ dx ~ \dudx ~
			\end{cases}} : \qquad \idu ~ \frac{1}{u} ~~ = ~~ \idx ~ e^{~ \alpha ~ x} \\ \\
		&\im ~~ ln \left| ~ u ~ \right| ~ + ~ C_1 ~~ = ~~ \idx ~ e^{~ \alpha ~ x} \qquad, ~~ C_1 ~ \in ~ \mathbb{K} \\ \\
		&\im ~~ { \begin{cases}
			~ \alpha ~ \neq ~ 0 ~
			\end{cases} } : \qquad ln \left| ~ y(x) ~ \right| ~ + ~ C_1 ~~ = ~~ \frac{ e^{~ \alpha ~ x} }{ \alpha } ~ + ~ \frac{C_2}{\alpha} \qquad, ~~ C_2 ~ \in ~ \mathbb{K} \\ \\
		&\im ~~ ln \left| ~ y(x) ~ \right| ~~ = ~~ \frac{ e^{~ \alpha ~ x} }{ \alpha } ~ + ~ \frac{C_2}{\alpha} ~ - ~ C_1 \\ \\
		&\im ~~ {\begin{cases}
			~ C_{(1)} \left( \alpha \right) ~ := ~ C_2 ~ - ~ C_1 ~~ \in ~ \mathbb{K} ~
			\end{cases}} : \qquad ln \left| ~ y(x) ~ \right| ~~ = ~~ \frac{ e^{~ \alpha ~ x} }{ \alpha } ~ + ~ C_{(1)} \left( \alpha \right) \\ \\
		&\im ~~ e^{~ ln \left| ~ y(x) ~ \right|} ~~ = ~~ e^{~ \frac{ e^{~ \alpha ~ x} }{ \alpha } ~ + ~  C_{(1)} \left( \alpha \right)} \\ \\
		&\im ~~ \left| ~ y(x) ~ \right| ~~ = ~~ e^{~ \frac{ e^{~ \alpha ~ x} }{ \alpha } ~ + ~  C_{(1)} \left( \alpha \right)} \\ \\
		&\im ~~ y(x) ~~ = ~~ \pm ~ e^{~ \frac{ e^{~ \alpha ~ x} }{ \alpha } ~ + ~  C_{(1)} \left( \alpha \right)}
	\end{aligned}$
	
	\newpage
	
	$\begin{aligned}[t]
		&\im ~~ y(x) ~~ = ~~ \pm ~ e^{~ \frac{ e^{~ \alpha ~ x} }{ \alpha } } ~ e^{ ~ C_{(1)} \left( \alpha \right) } \\ \\
		&\im ~~ {\begin{cases}
			~ C_{(2)} \left( \alpha \right) ~ := ~ e^{ C_{(1)} \left( \alpha \right) } ~~ \in ~ \mathbb{K} ~
		\end{cases}} : \qquad y(x) ~~ = ~~ \pm ~ e^{~ \frac{ e^{~ \alpha ~ x} }{ \alpha } } ~ C_{(2)} \left( \alpha \right) \\ \\
		&\im ~~ y(x) ~~ = ~~ \pm ~  C_{(2)} \left( \alpha \right) ~ e^{~ \frac{ e^{~ \alpha ~ x} }{ \alpha } } \\ \\ \\ % Anderer Fall
		&\im ~~ {\begin{cases}
			~ \alpha ~ = ~ 0 ~
		\end{cases}} : \qquad ln ~ \left| ~ y(x) ~ \right| ~~ = ~~ \idx ~ 1 \\ \\
		&\im ~~ ln ~ \left| ~ y(x) ~ \right| ~ + ~ C_1 ~~ = ~~ x ~ + ~ C_3 \qquad, ~~ C_3 ~ \in ~ \mathbb{K} \\ \\
		&\im ~~ ln ~ \left| ~ y(x) ~ \right| ~~ = ~~ x ~ + ~ C_3 ~ - ~ C_1 \\ \\
		&\im ~~ {\begin{cases}
			~ C_4 ~ := ~ C_3 ~ - ~ C_1 ~~ \in ~ \mathbb{K} ~
		\end{cases}} : \qquad ln ~ \left| ~ y(x) ~ \right| ~~ = ~~ x ~ + ~ C_4 \\ \\
		&\im ~~ e^{~ ln ~ \left| ~ y(x) ~ \right|} ~~ = ~~ e^{~ x ~ + ~ C_4} \\ \\
		&\im ~~ \left| ~ y(x) ~ \right| ~~ = ~~ e^{~ x ~ + ~ C_4} \\ \\
		&\im ~~ y(x) ~~ = ~~ \pm ~ e^{~ x ~ + ~ C_4} \\ \\
		&\im ~~ y(x) ~~ = ~~ \pm ~ e^{~ x} ~ e^{~ C_4} \\ \\
		&\im ~~ {\begin{cases}
			~ C_5 ~ := ~ e^{C_4} ~~ \in ~ \mathbb{K} ~
		\end{cases}} : \qquad y(x) ~~ = ~~ \pm ~ e^{~ x} ~ C_5 \\ \\
		&\im ~~ y(x) ~~ = ~~ \pm ~ C_5 ~ e^{~ x} \\ \\
	\end{aligned}$
	
	\newpage
	
	
	~\\	
	~\\
	
	$\begin{aligned}
	&\Rightarrow \qquad \dydx ~ = ~ e^{~ \alpha ~ x} ~ y(x) \\ \\
	&\Rightarrow ~~ {\begin{cases}
		~ y \left( x \right) ~ \neq ~ 0 ~ : & y \left( x \right) ~ = {\begin{cases}
			~ \alpha ~ \neq ~ 0 ~ : & \pm ~ C ~ e^{~ \frac{ e^{~ \alpha ~ x} }{ \alpha } } \\ \\
			~ \alpha ~ = ~ 0 ~ : & \pm ~ C ~ e^{~ x}
			\end{cases}} \\ \\
		~ y \left( x \right) ~ = ~ 0
		\end{cases}} \qquad, ~~ C ~ \in ~ \mathbb{K}
	\end{aligned}$
	
	\newpage
	
	\setcounter{tc}{0}
	
	\begin{longtable}{l@{\hspace{3em}}l}
		
		\itc & Trennung der Variablen ~ $y(x)$ ~ und ~ $x$. \\ \\
		\itc & ... \\
		
	\end{longtable}
	
	~\\
	~\\
	
	(TODO) Probe: ~ $y(x) ~ = ~ \pm ~ ...$ ~ :
	
	~\\
	
	$\begin{aligned}[t]
		\ddx ~ y(x) ~~ ...
	\end{aligned}$
	
	~\\
	
	Notiz: Es macht Sinn, Integrationskonstanten immer so lange wie möglich zu behalten? Ein Teil der Aufgabe(n) hat wohl eindringlich darauf abgezielt, daran zu erinnern, dass beim Integrieren auf Sonderfälle von gegebenen Parametern geachtet werden muss!
	
	
	
	

\end{enumerate}







% !TeX encoding = UTF-8



\newpage


\chapter*{Übungsblatt 3}
\addcontentsline{toc}{chapter}{3}

\newpage

\section*{\underline{Fragen zu den Aufgaben oder Allgemeinem}}
\addcontentsline{toc}{subsubsection}{Fragen}

\newpage




\section*{Aufgabe 5}
%\addcontentsline{toc}{section}{5}

~\\

\subsection*{5, a)}
\addcontentsline{toc}{section}{5, a)}

~\\

%	Eine Ameise befindet sich zum Zeitpunkt ~ $t ~ = ~ 0$ ~ am Ort ~ $x_0 ~ \geq ~ 0$ ~ eines Gummibandes, das bei ~ $x ~ = ~ 0$ ~ eingespannt ist. Die Länge des Gummibandes ist ~ $L(t) ~ = ~ L_0 ~ + ~ v_G ~ t$ ~ , d.h. es wird mit der (konstanten) Geschwindigkeit ~ $v_G$ ~ gedehnt. Die Ameise läuft mit Geschwindigkeit ~ $v_A$ ~ auf das Ende des Gummibandes zu. Sonstige Parameter, wie z.B. die Lebensdauer der Ameise oder die Zerreißlänge des Gummibandes werden auf ~ $\infty$ ~ gesetzt. \\
%	
%	a) Verifizieren Sie, dass der im Intervall ~ $\left[ t, ~ t ~ + ~ dt \right]$ ~ zurückgelegte Weg der Ameise ~ $dx ~ = ~ v_A ~ dt ~ + ~ v_G ~ \frac{ x(t) }{ L(t) } ~ dt$ ~ ist. \\
%	Hinweis: Betrachten Sie ~ $r(t) ~ = ~ \frac{ x(t) }{ L(t) }$ ~ und drücken Sie ~ $\dot{r}$ ~ durch ~ $L_0, ~ v_G$ ~ und ~ $v_A$ ~ aus. \\
%	
%	b) Berechnen Sie ~ $r(t)$ ~. Achten Sie dabei auf die Anfangsbedingung ~ $r(0) ~ = ~ \frac{x_0}{L_0}$ ~. Geben Sie die Zeit ~ $T$ ~ an, zu der die Ameise den Endpunkt ~ $x ~ = ~ L$ ~ erreicht hat. \\
%	
%	c) ...
%	
	
	% --- FALSCH
	% Das war Falsch: Aufgabe nicht verstanden! Habe das
	% was gezeigt werden sollte eingesetzt und mich im Kreis gedreht!!!
	% Spielchen: Hier kann x(t) oder v_G doch 0 werden, oder? Ja?
%	$\begin{aligned}[t]
%	\dot{r} ~~ &= ~~ \ddt ~ \frac{x(t)}{L(t)} \\ \\
%	&= ~~ \frac{ \ddt ~ x(t) ~ L(t) ~ - ~ x(t) ~ \ddt ~ L(t) }{ L(t)^2 } \\ \\
%	&= ~~ \frac{ \dot{x} ~ L(t) ~ - ~ x(t) ~ \ddt ~ \left( L_0 + v_G ~ t \right) }{ L(t)^2 } \\ \\
%	&= ~~ \frac{ \dot{x} ~ L(t) ~ - ~ x(t) ~ v_G }{ L(t)^2 } \\ \\
%	&= ~~ \frac{ \dot{x} }{ L(t) } ~ - ~ \frac{ x(t) ~ v_G }{ L(t)^2 } \\ \\
%	&= ~~ \frac{ \dot{x} }{ L(t) } ~ - ~ \frac{ x(t) ~ v_G }{ x(t) ~ v_G } ~ \frac{ x(t) ~ v_G }{ L(t)^2 } \\ \\
%	&= ~~ \frac{ \dot{x} }{ L(t) } ~ - ~ \frac{ 1 }{ x(t) ~ v_G } ~ \frac{ \left( x(t) ~ v_G \right)^{2} }{ L(t)^2 } \\ \\
%	&= ~~ \frac{ \dot{x} }{ L(t) } ~ - ~ \frac{ 1 }{ x(t) ~ v_G } ~ \left( \frac{  x(t) ~ v_G  }{ L(t) } \right)^{2} \\ \\
%	&= ~~ \frac{ \dot{x} }{ L(t) } ~ - ~ \frac{ 1 }{ x(t) ~ v_G } ~ \left( 0 ~ - ~ \frac{  x(t) ~ v_G  }{ L(t) } \right)^{2} \\ \\
%	&= ~~ \frac{ \dot{x} }{ L(t) } ~ - ~ \frac{ 1 }{ x(t) ~ v_G } ~ \left( - ~ \frac{dx}{dt} ~ + ~ \frac{dx}{dt} ~ - ~ \frac{  x(t) ~ v_G  }{ L(t) } \right)^{2} \\ \\
%	&= ~~ \frac{ \dot{x} }{ L(t) } ~ - ~ \frac{ 1 }{ x(t) ~ v_G } ~ \left( - ~ \frac{dx}{dt} ~ + ~ v_A \right)^{2} \\ \\
%	\end{aligned}$
% --- ENDE FALSCH ---

	\setcounter{tc}{0}

	~\\
	~\\

	$\begin{aligned}[t]
	\dot{r} ~~ &= ~~ \ddt ~ \frac{x(t)}{L(t)} \\ \\
	&= ~~ \frac{ \ddt ~ x(t) ~ L(t) ~ - ~ x(t) ~ \ddt ~ L(t) }{ L(t)^2 } \\ \\
	&= ~~ \frac{ \dot{x} ~ L(t) ~ - ~ x(t) ~ \ddt ~ \left( L_0 + v_G ~ t \right) }{ L(t)^2 } \\ \\
	&= ~~ \frac{ \dot{x} ~ L(t) ~ - ~ x(t) ~ v_G }{ L(t)^2 } \\ \\
	\end{aligned}$
	
	~\\
	
	\setcounter{tc}{0}
	
\begin{minipage}{0pt} % LATEX Trick, Besserer Weg?
\begin{flalign*}
%
% \im & & <left_expression> &= <right_expression>
%
\im \qquad & & \dot{r} ~ L(t)^2 ~~ &= ~~ \dot{x} ~ L(t) ~ - ~ x(t) ~ v_G \\ \\
%
\im \qquad & & \dot{r} ~ L(t)^2 ~ + ~ x(t) ~ v_G ~~ &= ~~ \dot{x} ~ L(t) \\ \\
%
\im \qquad & & \dot{r} ~ L(t) ~ + ~ \frac{ x(t) }{ L(t) } ~ v_G ~~ &= ~~ \dot{x} \\ \\
%
\im \qquad & & \dot{r} ~ L(t) ~ + ~ r(t) ~ v_G ~~ &= ~~ \dot{x} \\ \\
%
\im \qquad & & \dot{r} ~ L_0 ~ + ~ \dot{r} ~ v_G ~ t ~ + ~ r(t) ~ v_G ~~ &= ~~ \dot{x} &
%
\end{flalign*}
\end{minipage}








% % % % % % % % % % % % % % % % % % % % % % % % %


\newpage


\section*{Aufgabe 6}
%\addcontentsline{toc}{section}{6}

~\\

\subsection*{6, a)}
\addcontentsline{toc}{section}{6, a)}

~\\

\setcounter{tc}{0}
	
~\\ \begin{align*}
%
\frac{ 1 }{ \alpha ~ v ~ + ~ \beta ~ v^2 } ~~ &\eqs ~~ \frac{ 1 }{ \left( \alpha ~ + ~ \beta ~ v \right) ~ v } \\ \\
%
&\eqs ~~ \frac{ 1 }{ \left( \alpha ~ + ~ \beta ~ v \right) ~ v } \\ \\
%
&\eqs ~~ \frac{ 1 }{ \beta ~ \left( \frac{\alpha}{\beta} ~ + ~ v \right) ~ v } \qquad, ~~ \beta ~ \neq ~ 0  \\ \\
%
&\eqs ~~ \frac{ 1 }{ \beta ~ \left( ~ v ~ - ~ \left( -\frac{\alpha}{\beta} \right) ~ \right) ~ \left( v ~ - ~ 0 \right) }  \\ \\
%
\end{align*}

\setcounter{tc}{0}

\begin{minipage}{0pt}
	\begin{flalign*}
	%
	\im \qquad & & \frac{ 1 }{ \beta ~ \left( ~ v ~ + ~ \frac{\alpha}{\beta} ~ \right) ~ v } ~~ &= ~~ \frac{ K_1 }{ v } ~ + ~ \frac{ K_2 }{ v ~ + ~ \frac{\alpha}{\beta} } \qquad, ~~ K_1, ~ K_2 ~ \in \mathbb{R} \\ \\
	%
	\im \qquad & & 1 ~~ &= ~~ K_1 ~ \beta ~ \left( v ~ + ~ \frac{\alpha}{\beta} \right) ~ + ~ K_2 ~ \beta ~ v \\ \\
	%
	\end{flalign*}
\end{minipage}

\begin{minipage}{0pt}
	\begin{flalign*}
	\im \qquad & & ~~ {\left\{\begin{aligned} \setcounter{tct}{0}
			%
			~ & v ~ = ~ 0: \quad & & & 1 ~~ &= ~~ K_1 ~ \beta ~ \frac{\alpha}{\beta} ~ + ~ 0 \\ \\
			~ & &\imt \qquad & & 1 ~~ &= ~~ K_1 ~ \alpha \\ \\
			~ & &\imt \qquad & & {\begin{cases} \alpha ~ \neq ~\ 0 ~ \end{cases}} : \quad \frac{1}{\alpha} ~~ &= ~~ K_1
			%
		\end{aligned}\right.}  \\ \\
	%
	\end{flalign*}
\end{minipage}

\begin{minipage}{0pt}
	\begin{flalign*}
	\im \qquad & & ~~ {\left\{\begin{aligned} \setcounter{tct}{0}
		%
		~ & v ~ = ~ -\frac{\alpha}{\beta}: \quad & & & 1 ~~ &= ~~ 0 ~ + ~ K_2 ~ \beta ~ \left( -\frac{\alpha}{\beta} \right)  \\ \\
		~ & &\imt \qquad & & 1 ~~ &= ~~ K_2 ~ \left( -\alpha \right) \\ \\
		~ & &\imt \qquad & & {\begin{cases} \alpha ~ \neq ~\ 0 ~ \end{cases}} : \quad -\frac{1}{\alpha} ~~ &= ~~ K_2
		%
		\end{aligned}\right.}  \\ \\
	%
	\end{flalign*}
\end{minipage}

\begin{minipage}{0pt}
	\begin{flalign*}
	%
	\im \qquad & & \frac{1}{\alpha ~ v ~ + ~ \beta ~ v^2} ~~ &= ~~ \frac{1}{\alpha} ~ \frac{1}{v} ~ - ~ \frac{1}{\alpha} ~ \frac{1}{v ~ + ~ \frac{\alpha}{\beta}} \\ \\
	%
	\im \qquad & & &= ~~ \frac{1}{\alpha ~ v} ~ - ~ \frac{1}{\alpha ~ \left(v ~ + ~ \frac{\alpha}{\beta} \right)} \\ \\ \\
	%
	\im \qquad & & &= ~~ {\begin{cases} \beta ~ = ~ 0 \end{cases}} : \quad \frac{1}{\alpha ~ v ~ + ~ \beta ~ v^2} ~~ = ~~ \frac{1}{\alpha ~ v}
	%
	\end{flalign*}
\end{minipage}

~\\
~\\

\newpage
	
	\underline{Probe:} \setcounter{tc}{0}
	
	~\\
	
	\begin{minipage}{0pt}
		\begin{flalign*}
		%
		\im \qquad & & \frac{1}{\alpha ~ v ~ + ~ \beta ~ v^2} ~~ &= ~~ \frac{1}{\alpha ~ v} ~ - ~ \frac{1}{\alpha ~ \left(v ~ + ~ \frac{\alpha}{\beta} \right)} \\ \\
		%
		\end{flalign*}
	\end{minipage}
	
	\begin{minipage}{0pt}
		\begin{flalign*}
		\im \qquad & & {\left\{
		\begin{aligned}
		& \quad HN ~ := ~ \left( \alpha ~ v ~ + ~ \beta ~ v^2 \right) ~ \alpha ~ v ~ \left( v ~ + ~ \frac{\alpha}{\beta} \right) ~ : \\ \\
		& {\begin{aligned}
		%
		& & \quad \frac{\alpha ~ \left( v ~ + ~ \frac{\alpha}{\beta} \right) ~ v}{ HN } ~~ &= ~~ \frac{ \alpha ~ \left( v ~ + ~ \frac{\alpha}{\beta} \right) ~ v }{ HN } ~ + ~ \frac{v ~ \left( \alpha ~ v ~ + ~ \beta ~ v^2 \right)}{HN} \\ \\
		%
		\quad \imt & & \quad 0 ~~ &= ~~ {\begin{aligned}[t]
			& \alpha ~ \left( v ~ + ~ \frac{\alpha}{\beta} \right) ~ v \\ \\
			&+ ~ v ~ \left( \alpha ~ v ~ + ~ \beta ~ v^2 \right) \\ \\
			&- ~ \alpha ~ \left( v ~ + ~ \frac{\alpha}{\beta} \right) ~ v
			\end{aligned}} \\ \\
		%
		\quad \imt & & \quad  0 ~~ &= ~~ {\begin{aligned}[t]
			& \alpha ~ v^2 ~ + ~ \frac{\alpha^2 ~ v}{\beta} ~ + ~ \beta ~ v^3 ~ + ~ \alpha^2 ~ v \\ \\
			&- ~ \alpha^2 ~ v ~ - ~ \beta ~ v^3 \\ \\
			&- ~ \alpha ~ v^2 ~ - ~ \frac{\alpha^2 ~ v}{\beta}
			\end{aligned}} \\ \\
		%
		\quad \imt & & \quad  0 ~~ &= ~~ 0
		%
		\end{aligned}} \end{aligned} \right.}
		%
		\end{flalign*}
	\end{minipage}

%Notiz: Wenn der Grad des Zählers 0 ist und der Grad des Nenners 1, so steht die PBZ ja schon dort. Wenn man geschickt ist, kann man bei der Probe einen Term weglassen. Dieser kommt (wenn man nur bestimmte Umformungen durchführt, Welche?) dann am Ende raus. Hier würde man z.B.  $\frac{1}{\alpha ~ v ~ + ~ \beta ~ v^2} = - ~ \frac{1}{\alpha} ~ \frac{1}{v ~ + ~ \frac{\alpha}{\beta}}$ prüfen und $\frac{1}{\alpha ~ v}$ sollte rauskommen (Achtung, nur spezielle Umformungen). Hauptnenner HN (nicht ausgeschrieben).





\newpage
	

\subsection*{6, b)}
\addcontentsline{toc}{section}{6, b)}
	
\begin{flalign*}
	&\quad \qquad \dot{v} ~~ = ~~ -\alpha ~ v ~ - ~ \beta ~ v^2 \\ \\
	&\im \qquad \dot{v} ~ \frac{1}{\alpha ~ v ~ + ~ \beta ~ v^2} ~~ = ~~ -1 \\ \\
	&\im \qquad \dot{v} ~ \left( \frac{1}{\alpha ~ v} ~ - ~ \frac{1}{\alpha ~ \left(v ~ + ~ \frac{\alpha}{\beta} \right)} \right) ~~ = ~~ -1 \\ \\
	&\im \qquad \dot{v} ~ \frac{1}{\alpha ~ v} ~ - ~ \dot{v} ~ \frac{1}{\alpha ~ \left(v ~ + ~ \frac{\alpha}{\beta} \right)} ~~ = ~~ -1 \\ \\
	&\im \qquad \frac{1}{\alpha} ~\left( ~ \dot{v} ~ \frac{1}{v} ~ - ~ \dot{v} ~ \frac{1}{v ~ + ~ \frac{\alpha}{\beta}} ~ \right) ~~ = ~~ -1 \\ \\
	&\im \qquad \frac{1}{\alpha} ~\left( ~ \idt ~ \dot{v} ~ \frac{1}{v} ~ - ~ \idt ~ \dot{v} ~ \frac{1}{v ~ + ~ \frac{\alpha}{\beta}} ~ \right) ~~ = ~~ \idt ~ (-1) \\ \\
	&\im \qquad \frac{1}{\alpha} ~\left( ~ ln ~ \left| ~ v ~ \right| ~ + ~ C_1 ~ - ~ ln ~ \left| ~ v ~ + ~ \frac{\alpha}{\beta} ~ \right| ~ + ~ C_2 ~ \right) ~~ = ~~ -t ~ + ~ C_3 \\ \\
	&\im \qquad \frac{1}{\alpha} ~\left( ~ ln ~ \left| ~ \frac{v}{v ~ + ~ \frac{\alpha}{\beta}} ~ \right| ~ + ~ C_4 ~ \right) ~~ = ~~ -t ~ + ~ C_3 \\ \\
	&\im \qquad ln ~ \left| ~ \frac{v}{v ~ + ~ \frac{\alpha}{\beta}} ~ \right| ~ + ~ C_4 ~~ = ~~ - ~ \alpha ~ t ~ + ~ C_5 \\ \\
	&\im \qquad ln ~ \left| ~ \frac{v}{v ~ + ~ \frac{\alpha}{\beta}} ~ \right| ~~ = ~~ - ~ \alpha ~ t ~ + ~ C_6 \\ \\
	&\im \qquad e^{ln ~ \left| ~ \frac{v}{v ~ + ~ \frac{\alpha}{\beta}} ~ \right|} ~~ = ~~ e^{~ - ~ \alpha ~ t ~ + ~ C_6}
\end{flalign*}

\newpage

\begin{flalign*}
	&\im \qquad \left| ~ \frac{v}{v ~ + ~ \frac{\alpha}{\beta}} ~ \right| ~~ = ~~ e^{~ - ~ \alpha ~ t ~ + ~ C_6} \\ \\
	&\im \qquad \frac{v}{v ~ + ~ \frac{\alpha}{\beta}} ~~ = ~~ \pm ~ e^{~ - ~ \alpha ~ t ~ + ~ C_6} \\ \\
\end{flalign*}

~\\

Weiter? Vermutlich habe ich mich verrechnet. Allerdings ist mir auch unklar, ab wann ich die Integrationskonstante und welche ich durch $v_0$ ersetzen soll (oder alle zusammenfassen und das Resultat ersetzen?). Wie weit vereinfachen? Ich würde es so machen: in dem Moment wo links $v(t)$ steht wird noch ersetzt und fertig. Wie ist das mit dem Zeichnen? Ich nehme an von Hand. Mit $e$ ist das eben nicht ganz so einfach. Wie darf ich runden (3?)? Ich könnte jetzt noch eine v-P-Division machen und bekäme 1 plus den Rest. Danach stört mich leider der Ausdruck mit t.

Zudem ist mir der Lösungsansatz (Mitschrieb) unklar. Da in dieser nur ein Summand angegeben ist. Ich wollte diese Formel eigtl. verwenden.
% Überlege Dir die allgemeine Form von \dot{v} ~ 1 / (z (v) )



\newpage
	
\subsection*{6, c)}
\addcontentsline{toc}{section}{6, c)}



% !TeX encoding = UTF-8



\chapter*{Übungsblatt 4}
\addcontentsline{toc}{chapter}{4}

\newpage

\section*{\underline{Fragen zu den Aufgaben oder Allgemeinem}}
\addcontentsline{toc}{subsubsection}{Fragen}


~\\

\begin{enumerate}
	
	\item Wann macht es Sinn, (eine) Integrationskonstante(n) erst zum Schluss zu wählen? \\
	
	\begin{tabularx}{0.88\textwidth}{lX}
		$\bullet$ & ...
	\end{tabularx}
	
	~\\
	
	\item Wann und wo macht es Sinn, (eine) Integrationskonstante(n) nicht erst zum Schluss zu wählen? \\
	
	\begin{tabularx}{0.88\textwidth}{lX}
		$\bullet$ & ...
	\end{tabularx}
	
	~\\
	
	\item Wann sind Fragen nach einem Ausdruck, z.B. ~ $y(x)$ ~ , durch dessen explizite Angabe, bzw. durch dessen implizite Angabe zu beantworten? \\
	
	\begin{tabularx}{0.88\textwidth}{lX}
		$\bullet$ & ...
	\end{tabularx}
	
\end{enumerate}	



\newpage


%
% Aufgabe 7
%


\section*{Aufgabe 7}
%\addcontentsline{toc}{section}{7}

~\\


\subsection*{7, a)}
\addcontentsline{toc}{section}{7, a)}


\setcounter{tc}{0}

\begin{align*}
	\dot{v} ~~ &= ~~ \gamma ~ - ~ \alpha ~ v ~ - ~ \beta ~ v^2 \qquad \qquad \text{$\equiv$: (1)} \\ \\
   \dot{v} ~~ &= \quad F \left( ~ ~ v^2 ~ , ~ v ~ , ~ t ~ \right) ~ + ~ \gamma \\ \\
    \quad F \left( ~ \dot{v} ~ , ~ v^2 ~ , ~ v ~ , ~ t ~ \right) ~~ &\neq ~~ 0
\end{align*}

\hfill

Gegeben ist eine gewöhnliche, nicht-lineare, nicht-homogene DGL von erster Ordnung mit konstanten Koeffizienten in expliziter Form.

~\\

\begin{description}[leftmargin=*, labelsep=2em, itemsep=2em]
	
	\item[\textquotedblleft gewöhnliche\textquotedblright] \hfill \break
	
	Es gibt genau eine Variable ~ $t$ ~, nach der abgeleitet wird.

	\item[\textquotedblleft nicht-lineare\textquotedblright] \hfill \break
	
	Es kommt eine nicht-lineare Funktion in ~ $v$ ~, hier ~ $v^2$ ~ vor.
	
	\item[\textquotedblleft nicht-homogene\textquotedblright] \hfill \break
	
	Der von ~ $t$ ~ unabhängige Ausdruck ist nicht ~ $0$ ~ .

	\item[\textquotedblleft von erster Ordnung\textquotedblright] \hfill \break
	
	Von allen Ableitungen der durch ~ $v$ ~ beschriebenen Funktion ist hier ~ ${\dot{v}}$ ~ die höchste und die Anzahl ihrer Ableitungen ist ~ $1$ ~.
		
	\item[\textquotedblleft mit konstanten Koeffiziebten\textquotedblright] \hfill \break
	
	Die Koeffizienten vor den durch ~ $v$ ~ beschriebenen Funktionen sind alle konstant.
	
	\item[\textquotedblleft in expliziter Form\textquotedblright] \hfill \break
	
	Auf einer Seite der Differentialgleichung steht die höchste Ableitung.
	
\end{description}



~\\
~\\


\subsection*{7, b)}
\addcontentsline{toc}{section}{7, b)}

~\\
~\\

\subsection*{7, c)}
\addcontentsline{toc}{section}{7, c)}


~\\
~\\

\subsection*{7, d)}
\addcontentsline{toc}{section}{7, d)}




%
% Aufgabe 8
%


\newpage


\section*{Aufgabe 8}
%\addcontentsline{toc}{section}{7}

~\\


\subsection*{8, a)}
\addcontentsline{toc}{section}{8, a)}


~\\


\subsection*{8, b)}
\addcontentsline{toc}{section}{8, b)}
%% !TeX encoding = UTF-8



\newpage



\chapter{~}
\section{~}

~\\






\section{~}

~\\
%% !TeX encoding = UTF-8



\newpage



\chapter{~}
\section{~}

~\\






\section{~}

~\\
%% !TeX encoding = UTF-8



\newpage



\chapter{~}
\section{~}

~\\






\section{~}

~\\
%% !TeX encoding = UTF-8



\newpage



\chapter{~}
\section{~}

~\\






\section{~}

~\\
%% !TeX encoding = UTF-8



\newpage



\chapter{~}
\section{~}

~\\






\section{~}

~\\

\newpage

\setcounter{chapter}{0}

\chapter*{Präsenzübungen}
\addcontentsline{toc}{chapter}{Präsenzübungen}

% !TeX encoding = UTF-8


\chapter{~}


\section{~}


\newpage


\section{~}
% !TeX encoding = UTF-8


\chapter{~}


\section{~}


\newpage


\section{~}
% !TeX encoding = UTF-8


\chapter*{Präsenzübung 3}
\addcontentsline{toc}{chapter}{3}


~\\


\section*{Aufgabe 1}


~\\


\subsection*{1, a)}
\addcontentsline{toc}{section}{1, a)}

~\\

\[ \tcbhighmath[boxrule=2pt]{ \int_{x_0}^{x_1} ~ dt ~ \frac{\dot{y}}{y} ~~ = ~~ ln ~ \left| ~ y(x_1) ~ \right| ~ - ~ ln ~ \left| ~ y(x_0) ~ \right| } \]

~\\


\begin{minipage}{0pt}
	\begin{flalign*}
	%
	\quad \qquad & & \dot{v} ~~ &= ~~ -f(t) ~ v \\ \\
	%
	\im \qquad & & \frac{\dot{v}}{v} ~~ &= ~~ -f(t) \\ \\
	%
	\im \qquad & & \int_{0}^{t} ~ dt ~ \frac{\dot{v}}{v} ~~ &= ~~ \underbrace{ -\int_{0}^{t} ~ dt ~ f(t) }_{=: ~ -F(t)} \\ \\
	%
	\im \qquad & & ln ~ \left| ~ v(t) ~ \right| ~ - ~ ln ~ \left| ~ v(0) ~ \right| ~~ &= ~~ -F(t) \\ \\
	%
	\im \qquad & & ln ~ \left| ~ \frac{v(t)}{v(0)} ~ \right| ~~ &= ~~ -F(t) \\ \\
	%
	\end{flalign*}
\end{minipage}

\begin{minipage}{0pt}
	\begin{flalign*}
	%
	\im \qquad & & e^{ ~ ln ~ \left| ~ \frac{v(t)}{v(0)} ~ \right|} ~~ &= ~~ e^{ ~ -F(t)} \\ \\
	%
	\im \qquad & & \left| ~ \frac{v(t)}{v(0)} ~ \right| ~~ &= ~~ e^{ ~ -F(t)} \\ \\
	%
	\im \qquad & & \frac{v(t)}{v(0)} ~~ &= ~~ \pm ~ e^{ ~ -F(t)} \\ \\
	%
	\im \qquad & & v(t) ~~ &= ~~ \pm ~ e^{ ~ -F(t)} ~ v(0) \\ \\
	%
	\im \qquad & & v(t) ~ - ~ v(0) ~~ &= ~~ \pm ~ e^{ ~ -F(t)} ~ v(0) ~ - ~ v(0) \\ \\
	%
	\im \qquad & & v(t) ~ - ~ v(0) ~~ &= ~~ \pm ~ \left( ~ e^{ ~ -F(t)} ~ - ~ 1 ~ \right) ~ C_1 \qquad, ~~ C_1 ~ := ~ v(0) ~ \in ~ \mathbb{K}
	%
	\end{flalign*}
\end{minipage}

~\\
~\\

\subsection*{1, b)}
\addcontentsline{toc}{section}{1, b)}

~\\

\setcounter{tc}{0}

\begin{minipage}{0pt}
	\begin{flalign*}
	%
	\quad \qquad & & {\begin{cases} ~ f(t) ~ = ~ \alpha ~ \end{cases}} : \qquad v(t) ~ - ~ v(0) ~~ &= ~~ \pm ~ \left( ~ e^{ ~ -\int_{0}^{t} ~ dt ~ \alpha } ~ - ~ 1 ~ \right) ~ C_1 \\ \\
	%
	\im \quad \qquad & & v(t) ~ - ~ v(0) ~~ &= ~~ \pm ~ \left( ~ e^{ ~ -\alpha ~ \left[ t \right]_{0}^{t} } ~ - ~ 1 ~ \right) ~ C_1 \\ \\
	%
	\im \quad \qquad & & v(t) ~ - ~ v(0) ~~ &= ~~ \pm ~ \left( ~ e^{ ~ -\alpha ~ t} ~ - ~ 1 ~ \right) ~ C_1 \\ \\
	%
	\im \quad \qquad & & v(t) ~~ &= ~~ C_2 \qquad, ~~ C_2 ~ \in ~ \mathbb{K}
	%
	\end{flalign*}
\end{minipage}


%~\\
%~\\

\newpage


\subsection*{1, c)}
\addcontentsline{toc}{section}{1, c)}

~\\

Gegeben ist eine gewöhnliche, nicht-lineare, homogene DGL, von erster Ordnung mit konstanten Koeffizienten in expliziter Form.

~\\

\setcounter{tc}{0}

\begin{minipage}{0pt}
	\begin{flalign*}
	%
	\quad \qquad & & \dydx ~~ &= ~~ a ~ y^2 ~ + ~ b ~ y \\ \\
	%
	\im \quad \qquad & & \dydx ~ \frac{1}{a ~ y^2 ~ + ~ b ~ y} ~~ &= ~~ 1 \\ \\
	%
	\end{flalign*}
\end{minipage}

~\\

\begin{minipage}{0pt}
	\begin{flalign*}
	\end{flalign*}
\end{minipage}







\newpage


%\section{~}
% !TeX encoding = UTF-8


\chapter*{Präsenzübung 4}
\addcontentsline{toc}{chapter}{3}


\newpage

\underline{Fragen zu den Aufgaben oder Allgemeinem:}

~\\

\begin{enumerate}
	
	\item Ist eine Integration über ~ $0$ ~ von der Integrationsvariable unabhängig, d.h. ~ $ \int ~ 0 ~ $ ~ ist für alle Integrationsvariablen aus beliebigen Mengen definiert? Unter welcher Einschränkung gilt es? \\
	
	\begin{tabularx}{0.88\textwidth}{lX}
		$\bullet$ & ...
	\end{tabularx}
	
	~\\
	
	\item Wenn ~ $ \int ~ 0 ~ $ ~ nicht für alle Integrationsvariablen aus beliebigen Mengen definiert ist, unter welcher Einschränkung ist es definiert? \\
	
	\begin{tabularx}{0.88\textwidth}{lX}
		$\bullet$ & ...
	\end{tabularx}
	
	~\\Notizen: Das müsste stimmen, wenn der Quantor frei wird. Einfach Mal die Def. aufschreiben.
	~\\
	
		
	\item Gibt es eine einfache Merkhilfe für Hyperbolicus-Funktionen, wie z.B. ~ $cosh ~ \varphi$ ~ oder ~ $sinh ~ \varphi$ ~ ? \\
		
	\begin{tabularx}{0.88\textwidth}{lX}
		$\bullet$ & ...
	\end{tabularx}
		
	~\\
	
	\item Wie ist der $\nabla$-Operator ohne Pünktchen definiert? \\
	
	\begin{tabularx}{0.88\textwidth}{lX}
		$\bullet$ & ...
	\end{tabularx}
	
	~\\

\end{enumerate}




\newpage


~\\


\section*{Aufgabe 1}


~\\


\subsection*{1, a)}
\addcontentsline{toc}{section}{1, a)}

~\\

Annahme: \qquad $\dot{\vec{r}}(t) ~ \in ~ \mathbb{R}^{ ~ n}$

~\\

\begin{align*}
	\dot{\vec{r}}(t) ~~ = ~~ \left( \begin{array}{c} v_x \\ 0 \\ -gt \end{array} \right)
\end{align*}

~\\

\begin{align*}
	v(t) ~~ = ~~ \left| \dot{\vec{r}}(t) \right| ~~ = ~~ \sqrt{ v_x^2 ~ + ~ \left( -g ~ t \right)^2 } ~~ = ~~ \sqrt{ v_x^2 ~ + ~ g^2 ~ t^2 }
\end{align*}

~\\

\begin{align*}
\ddot{\vec{r}}(t) ~~ = ~~ \left( \begin{array}{c} 0 \\ 0 \\ -g \end{array} \right)
\end{align*}

~\\

\begin{align*}
a(t) ~~ = ~~ \left| \ddot{\vec{r}}(t) \right| ~~ = ~~ \sqrt{  \left( -g \right)^2 } ~~ = ~~ \sqrt{ g^2 } ~~ = ~~ \left| g \right|
\end{align*}

~\\
~\\

\subsection*{1, b)}
\addcontentsline{toc}{section}{1, b)}

~\\


\begin{align*}
	a(t) ~~ &= ~~ \ddt ~ \left| \dot{\vec{r}}(t) \right| \\ \\
	&= ~~ \ddt ~ \sqrt{ \underbrace{ v_x^2 ~ + ~ g^2 ~ t^2 }_{=: ~ t'} } \\ \\
	&= ~~ \frac{d}{d ~ t'} ~ ( ~ t' ~ )^{\frac{1}{2}} ~~ \ddt ~ \left( v_x^2 ~ + ~ g^2 ~ t^2 \right) \\ \\
	&= ~~ \frac{1}{2} ~ ( ~ v_x^2 ~ + ~ g^2 ~ t^2 ~ )^{-\frac{1}{2}} ~~ g^2 ~ 2 ~ t \\ \\
	&= ~~ \frac{g^2 ~ t}{\sqrt{v_x^2 ~ + ~ g^2 ~ t^2}}
\end{align*}

~\\

\begin{align*}
	\vec{a}_{\parallel}(t) ~~ &= ~~ \dot{\vec{r}} ~ \frac{a(t)}{v(t)}
\end{align*}

~\\

\begin{align*}
	\vec{a}_{\perp}(t) ~~ &= ~~ \ddot{\vec{r}}(t) ~ - ~ a_{\parallel}(t)
\end{align*}

~\\

\begin{align*}
\left| ~ \vec{a}_{\parallel}(t) ~ \right| ~~ &= ~~ \left| ~ \dot{\vec{r}} ~ \frac{a(t)}{v(t)} ~ \right|
\end{align*}

~\\

\begin{align*}
\left| ~ \vec{a}_{\perp}(t) ~ \right| ~~ &= ~~ \left| ~ \ddot{\vec{r}}(t) ~ - ~ a_{\parallel}(t) ~ \right|
\end{align*}

~\\

\begin{align*} \setcounter{tc}{0}
&\qquad \vec{a}_{\parallel}(t) ~ \cdot ~ \vec{a}_{\perp}(t) \\ \\
&\eqs ~~ {\begin{cases} \quad f ~ := ~ \frac{g^2 ~ t}{\sqrt{v_x^2 ~ + ~ g^2 ~ t^2}} ~ \end{cases}} : \qquad \left( \begin{array}{c} v_x \\ 0 \\ -gt \end{array} \right) ~ f ~ \left( ~ \left( \begin{array}{c} 0 \\ 0 \\ -g \end{array} \right) ~ - ~ \left( \begin{array}{c} v_x \\ 0 \\ -gt \end{array} \right) ~ f ~ \right) \\ \\
%
&\eqs ~~ f ~ \left( g^2 ~ t ~ - ~ \underbrace{ \left( v_x^2 ~ + ~ g^2 ~ t^2 \right) ~ f}_{= ~ g^2 ~ t} \right) \\ \\
&\eqs ~~ f ~ 0 \\ \\
&\eqs ~~ 0
\end{align*}


~\\

%Dieselbe Aufgabe wurde vor ? Jahren in LA gestellt, mit der dringenden Empfehlung,
%Vektorberechnungen nur minimal und geschickt auszuführen, bis es sein muss und die Verträglichkeit bestimmter Operationen auszunutzen. Daran halte ich mich. Die Aufgabe ist korrekt gelöst. Das Berechnen, welches in der Aufgabe verlangt wird ist im puren funktionalen Sinn erfüllt.

% Merkregel für Ableitung und Integral von cos/sin:
%
% cos: Vor der Integration lernen wir die Differentation. (zeitlich)
%      Der Kosinusgraph steigt erst von der Achse ab,
%      seine Ableitung ist der negative Sinus: -sin. (zeitlich)
%      
%
% sin: Analog.
%
%
%      
% Wenn man sich jetzt noch ein Bildchen vom alten Graf Newton denkt, ... dann vergisst man das nie wieder.


\newpage


\section*{Aufgabe 1}

\hfill

\subsection*{2, a)}
\addcontentsline{toc}{section}{2, a)}

\hfill

\begin{minipage}{0pt} \setcounter{tc}{0}
	\begin{flalign*}
	%
	\quad \qquad & & \ddot{x} ~~ &= ~~ \omega ~ \dot{y} \\ \\
	%
	\im \qquad & & \idt ~ \ddt ~ \dot{x} ~~ &= ~~ \omega ~ \idt ~ \ddt ~ y \\ \\
	%
	\im \qquad & & \dot{x} ~ + ~ C_1 ~~ &= ~~ \omega ~ \left( ~ y ~ + ~ C_2 ~ \right) \qquad, ~~ C_1, ~ C_2 ~ \in ~ \mathbb{K} \\ \\
	%
	\im \qquad & & \dot{x} ~~ &= ~~ \omega ~ y ~ + ~ C_3 \qquad, ~~ C_3 ~ \in ~ \mathbb{K}
	%
	\end{flalign*}
\end{minipage}

~\\
~\\

\begin{minipage}{0pt} \setcounter{tc}{0}
	\begin{flalign*}
	%
	\quad \qquad & & \ddot{y} ~~ &= ~~ -\omega ~ \dot{x} \\ \\
	%
	\im \qquad & & \dot{y} ~~ &= ~~ -\omega ~ x ~ + ~ C_3 \qquad, ~~ C_3 ~ \in ~ \mathbb{K}
	%
	\end{flalign*}
\end{minipage}

~\\

\[ \tcbhighmath[boxrule=2pt]{ \int ~ 0 ~~ = ~~ C \qquad, ~~ C ~ \in ~ \mathbb{K} } \]

\hfill

\begin{minipage}{0pt} \setcounter{tc}{0}
	\begin{flalign*}
	%
	\quad \qquad & & \ddot{z} ~~ &= ~~ 0 \\ \\
	%
	\im \qquad & & \idt ~ \ddt ~ \dot{z} ~~ &= ~~ \idt ~ 0 \\ \\
	%
	\im \qquad & & \dot{z} ~ + ~ C_1 ~~ &= ~~ C_2 \qquad, ~~ C_1, C_2 ~ \in ~ \mathbb{K} \\ \\
	%
	\im \qquad & & \dot{z} ~~ &= ~~ C_3 \qquad, ~~ C_3 ~ \in ~ \mathbb{K}
	%
	\end{flalign*}
\end{minipage}




\subsection*{2, b)}
\addcontentsline{toc}{section}{2, b)}

\hfill




~\\
~\\



\newpage


\section*{Aufgabe 3}

\hfill

\subsection*{3, a)}
\addcontentsline{toc}{section}{2, c)}

\hfill


\[ \tcbhighmath[boxrule=2pt]{ cosh ~ \varphi ~~ = ~~ \frac{ e^{ ~ \varphi} ~ + ~ e^{ ~ -\varphi } }{2} } \]

\[ \tcbhighmath[boxrule=2pt]{ sinh ~ \varphi ~~ = ~~ \frac{ e^{ ~ \varphi} ~ - ~ e^{ ~ -\varphi } }{2} } \]

\hfill

\begin{align*}
	cosh ~ \varphi ~~ = ~~ \frac{ e^{ ~ \varphi} ~ + ~ e^{ ~ -\varphi } }{2}
\end{align*}

\begin{align*}
	sinh ~ \varphi ~~ = ~~ \frac{ e^{ ~ \varphi} ~ - ~ e^{ ~ -\varphi } }{2}
\end{align*}


\newpage


\subsection*{3, b)}
\addcontentsline{toc}{section}{2, b)}

\hfill

%\[ \tcbhighmath[boxrule=2pt]{ \vec{\nabla} ~ f ~ a ~~ = ~~ \frac{\partial}{\partial ~ a_i} ~ f ~ a } \]

\hfill



\begin{align*}
	\vec{\nabla} ~ \left| ~ \vec{r} ~ \right| ~~ &= ~~ \vec{\nabla} ~ \underbrace{ \sqrt{ ~ x^2 ~ + ~ y^2 ~ + ~ z^2 ~ } }_{=: ~ f} \\ \\
	&= ~~ \left( \arraycolsep=1.4pt\def\arraystretch{1.3}\begin{array}{c} \px ~ f \\ \py ~ f \\ \pz ~ f \end{array} \right) \\ \\
	&= ~~ \left( \begin{array}{c} \frac{1}{2} ~ \left( ~ x^2 ~ + ~ y^2 ~ + ~ z^2 ~ \right)^{ ~ -\frac{1}{2}} ~ 2 ~ x \\ ... \\ ... \end{array} \right) \\ \\
	&= ~~ \left( \arraycolsep=1.4pt\def\arraystretch{1.3}\begin{array}{c} \frac{x}{f} \\ \frac{y}{f} \\ \frac{z}{f} \end{array} \right)
\end{align*}

% matlab
%syms x y z
%>> f = sqrt(x^2+y^2+z^2)
%
%f =
%
%(x^2 + y^2 + z^2)^(1/2)
%
%>> gradient(f, [x,y,z])
%
%ans =
%
%x/(x^2 + y^2 + z^2)^(1/2)
%y/(x^2 + y^2 + z^2)^(1/2)
%z/(x^2 + y^2 + z^2)^(1/2)






% Tipps: In Klausur nur Folgepfeile, wenn nicht anders verlangt.
% Durchstreichen und noch einen Umformungsschritt einkritzeln ist in derselben Zeile
% oft schneller, als das Ganze nochmal neu zu schreiben (in einer neuen Zeile). Spart Klausur-Zeit!
	
\end{document}