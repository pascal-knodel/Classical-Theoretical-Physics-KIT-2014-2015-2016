% !TeX encoding = UTF-8



\chapter*{Übungsblatt 4}
\addcontentsline{toc}{chapter}{4}

\newpage

\section*{\underline{Fragen zu den Aufgaben oder Allgemeinem}}
\addcontentsline{toc}{subsubsection}{Fragen}


~\\

\begin{enumerate}
	
	\item Wann macht es Sinn, (eine) Integrationskonstante(n) erst zum Schluss zu wählen? \\
	
	\begin{tabularx}{0.88\textwidth}{lX}
		$\bullet$ & ...
	\end{tabularx}
	
	~\\
	
	\item Wann und wo macht es Sinn, (eine) Integrationskonstante(n) nicht erst zum Schluss zu wählen? \\
	
	\begin{tabularx}{0.88\textwidth}{lX}
		$\bullet$ & ...
	\end{tabularx}
	
	~\\
	
	\item Wann sind Fragen nach einem Ausdruck, z.B. ~ $y(x)$ ~ , durch dessen explizite Angabe, bzw. durch dessen implizite Angabe zu beantworten? \\
	
	\begin{tabularx}{0.88\textwidth}{lX}
		$\bullet$ & ...
	\end{tabularx}
	
\end{enumerate}	



\newpage


%
% Aufgabe 7
%


\section*{Aufgabe 7}
%\addcontentsline{toc}{section}{7}

~\\


\subsection*{7, a)}
\addcontentsline{toc}{section}{7, a)}


\setcounter{tc}{0}

\begin{align*}
	\dot{v} ~~ &= ~~ \gamma ~ - ~ \alpha ~ v ~ - ~ \beta ~ v^2 \qquad \qquad \text{$\equiv$: (1)} \\ \\
   \dot{v} ~~ &= \quad F \left( ~ ~ v^2 ~ , ~ v ~ , ~ t ~ \right) ~ + ~ \gamma \\ \\
    \quad F \left( ~ \dot{v} ~ , ~ v^2 ~ , ~ v ~ , ~ t ~ \right) ~~ &\neq ~~ 0
\end{align*}

\hfill

Gegeben ist eine gewöhnliche, nicht-lineare, nicht-homogene DGL von erster Ordnung mit konstanten Koeffizienten in expliziter Form.

~\\

\begin{description}[leftmargin=*, labelsep=2em, itemsep=2em]
	
	\item[\textquotedblleft gewöhnliche\textquotedblright] \hfill \break
	
	Es gibt genau eine Variable ~ $t$ ~, nach der abgeleitet wird.

	\item[\textquotedblleft nicht-lineare\textquotedblright] \hfill \break
	
	Es kommt eine nicht-lineare Funktion in ~ $v$ ~, hier ~ $v^2$ ~ vor.
	
	\item[\textquotedblleft nicht-homogene\textquotedblright] \hfill \break
	
	Der von ~ $t$ ~ unabhängige Ausdruck ist nicht ~ $0$ ~ .

	\item[\textquotedblleft von erster Ordnung\textquotedblright] \hfill \break
	
	Von allen Ableitungen der durch ~ $v$ ~ beschriebenen Funktion ist hier ~ ${\dot{v}}$ ~ die höchste und die Anzahl ihrer Ableitungen ist ~ $1$ ~.
		
	\item[\textquotedblleft mit konstanten Koeffiziebten\textquotedblright] \hfill \break
	
	Die Koeffizienten vor den durch ~ $v$ ~ beschriebenen Funktionen sind alle konstant.
	
	\item[\textquotedblleft in expliziter Form\textquotedblright] \hfill \break
	
	Auf einer Seite der Differentialgleichung steht die höchste Ableitung.
	
\end{description}



~\\
~\\


\subsection*{7, b)}
\addcontentsline{toc}{section}{7, b)}

~\\
~\\

\subsection*{7, c)}
\addcontentsline{toc}{section}{7, c)}


~\\
~\\

\subsection*{7, d)}
\addcontentsline{toc}{section}{7, d)}




%
% Aufgabe 8
%


\newpage


\section*{Aufgabe 8}
%\addcontentsline{toc}{section}{7}

~\\


\subsection*{8, a)}
\addcontentsline{toc}{section}{8, a)}


~\\


\subsection*{8, b)}
\addcontentsline{toc}{section}{8, b)}