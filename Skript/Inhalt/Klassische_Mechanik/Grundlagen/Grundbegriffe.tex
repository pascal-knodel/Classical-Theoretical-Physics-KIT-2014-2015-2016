% !TeX encoding = UTF-8



\subsection{Grundbegriffe}

~\\


\underline{Theoretische Physik:} Beschreibung physikalischer Gesetzmäßigkeiten in mathematischer Sprache als \underline{physikalische Theorie}. Eine Theorie muss \underline{deskriptiv} und \underline{prädiktiv} sein.\\

\begin{description}
	
	\item[Deskriptiv:] \hfill
	
	Experimentelle Daten müssen korrekt beschrieben werden.\\
	
	\item[Prädiktiv:] \hfill
	
	Eine Theorie muss Vorhersagen über neue Experimente machen.\\
	
\end{description}

\begin{flushleft}
	
	Insbesondere muss sie \underline{falsifizierbar} sein (s. u. A.: \textit{experimentum crucis}, Francis Bacon, Karl Popper). Naturwissenschaftliche Theorien werden erraten. Sie sind unbeweisbar. D.h. jede naturwissenschaftlichen Theorie fängt \underline{axiomatisch} an.
	
	\begin{description}
		
		\item[Axiom:] \hfill
		
		Nicht beweisbare Aussage (Annahme).\\
		
	\end{description}
	
	\hfill \break Durch Aussagen über Experimente werden Theorien \underline{induktiv} aufgestellt. Diese selbst liefern Aussagen über zukünftige Experimente. \break
	
	\begin{description}
		
		\item[Induktion:] \hfill
		
		Schließen vom Speziellen zum Allgemeinen:\\ Empirie (speziell) $\longrightarrow$ Theorie (allgemein).\\
		
		\item[Deduktion:] \hfill
		
		Schließen vom Allgemeinen zum Speziellen:\\ Theorie (allgemein) $\longrightarrow$ Empirie (speziell).\\
		
	\end{description}
	
	
\end{flushleft}



\newpage



\begin{flushleft}
	
	Bei der Auswahl aus mehreren Theorien, welche die Daten beschreiben, ist die einfachste zu wählen. Ein anderer Name für dieses Konzept ist (Ockhams) (Rasier-)Messer-Prinzip, von
	
	
	\begin{flalign*}
	\begin{rcases}
	\text{Wilhelm of} \\
	\text{Wilhelm von}
	\end{rcases} \text{Ockham (1285 - 1349) }
	\end{flalign*}
	
	\hfill \break Das Messer steht dafür, alles abzuschneiden was unnötig ist. Naturwissenschaftliche Theorien müssen allgemeingültig sein, d.h. immer und überall gelten.\\ \hfill \break
	
	\underline{Klassische Mechanik:} Physik bewegter makroskopischer Objekte (Körper) unter dem Einfluss von Kräften.\\ \hfill \break
	
	\begin{description}
		
		\item[\underline{Kinematik:}] Lehre über Bewegungen
		
		\item[\underline{Dynamik:}] Lehre über Kräfte
		
		\begin{itemize}
			
			\item Körper in Kraftfeldern
			
			\item Klassifizierung von Kräften
			
			\item Gravitation (Schwerkraft)
			
			\item Himmelsmechanik (Planetenbewegung)
			
		\end{itemize}
		
	\end{description}
	
\end{flushleft}	
